\documentclass[a4paper,10pt]{scrartcl}

% Inclusión de paquetes
\usepackage[utf8]{inputenc}
\usepackage[spanish]{babel}

\usepackage{dsfont}
\usepackage{amsmath}
\usepackage{amssymb}
\usepackage{enumerate}
\usepackage{verbatim}
\usepackage{amsthm}

%\usepackage{euler}

% Imágenes
\usepackage{graphicx}
\usepackage{float}
\usepackage{tikz}
\usetikzlibrary{arrows}

% Enlaces dentro del documento
\usepackage{hyperref}
\usepackage{cleveref}

% Definiciones
\theoremstyle{definition}
\newtheorem*{definition}{Definición}
%\newtheorem{mydefn}{Definición}
\newtheorem*{theorem*}{Teorema}
\newtheorem{theorem}{Teorema}
\newtheorem*{lemma*}{Lema}
\newtheorem{lemma}{Lema}
\newtheorem*{fact*}{Proposición}
\newtheorem{fact}{Proposición}
\newtheorem*{corollary*}{Corolario}
\newtheorem{corollary}{Corolario}
\newtheorem*{rmk*}{Nota}
\newtheorem*{eg}{Ejemplo}


\newcommand{\ktilde}{\widetilde{K}}
\newcommand{\dtilde}[1]{\widetilde{\widetilde{#1}}}
% Displaystyle por defecto
\everymath{\displaystyle}

% Comandos
\newcommand{\Referencia}[4]{\indent #1, \textbf{#2}. \textit{#3}, \textit{#4}.\\}
\renewcommand\refname{Referencias}
\renewcommand\contentsname{Contenidos}
\numberwithin{equation}{section}
\setlength{\parindent}{0cm} % Sin sangrías
\setlength{\parskip}{0.3cm}


\title{Aproximación discreta de la ecuación de Vlasov-Poisson}
\author{
	Ignacio Cordón
}
\date{}

\begin{document}
\maketitle
\begin{center}
    %\includegraphics[width=0.4\textwidth]{./imgs/by-nc-sa.png}
\end{center}
\tableofcontents
\pagebreak

El presente trabajo está principalmente basado en el artículo \cite{Schaeffer}. Introducimos primeramente las ecuaciones de Vlasov y de Vlasov-Poisson. A continuación proporcionamos una serie de preliminares para ecuaciones de Vlasov-Poisson con condición inicial esféricamente simétrica y de soporte compacto, para finalmente introducir un método numérico para aproximar las soluciones que hace uso de dicha simetría esférica.

\section{Introducción}
En la naturaleza podemos encontrar 4 estados de la materia: sólido, líquido, gaseoso y plasma. Este último se alcanza con grandes temperaturas a las que los electrones de los átomos abandonan la órbita alrededor del núcleo al que pertenecen, dando lugar a una mezcla de iones, electrones y particulas gobernadas por fuerzas electromagnéticas. Este estado de la materia cobra especial relevancia en procesos como la fusión termonuclear, para la que se necesitan grandes temperaturas.  

La \textit{ecuación de Vlasov} es la encargada de modelar la evolución de las partículas de un plasma desde un punto de vista estadístico-físico:

\begin{equation}
 \frac{\partial f}{\partial t} + v \frac{\partial f}{\partial x} + \frac{q}{m} (E + v\times B) \cdot \frac{\partial f}{\partial v} = 0
\end{equation}

donde:

\begin{itemize}
 \item $f:\mathbb{R}^3 \times \mathbb{R}^3 \times \mathbb{R}^{+}_0 \rightarrow \mathbb{R}$ es una función de densidad asociada a la probabilidad de tener una partícula con posición $x \in \mathbb{R}^3$ y velocidad $v \in \mathbb{R}^3$ en un instante de tiempo $t$. 
 \item $E$ representa el campo eléctrico.
 \item $B$ representa el campo de inducción magnética.
 \item $F(x,t) = \frac{q}{m}(E + v\times B)$ es la fuerza de Lorentz.
\end{itemize}

El campo electromagnético viene dado por las ecuaciones de Maxwell:

\begin{align*}
 -\frac{1}{c^2} \frac{\partial E}{\partial t} + \nabla \times B = \mu_0 J\\
 \frac{\partial B}{\partial t} + \nabla \times E = 0\\
 \nabla \cdot E = \frac{\rho}{\varepsilon_0}\\
 \nabla \cdot B = 0
\end{align*}

donde 
\begin{align*}
\rho(x,t) &= q \int_{\mathbb{R}^3} f(x,v,t) dv\\
J(x,t) &= q \int_{\mathbb{R}^3} f(x,v,t) v dv
\end{align*}

Cuando el campo magnético es despreciable en comparación con el eléctrico, las ecuaciones de Maxwell quedan reducidas a:

\begin{equation*}
 \nabla \times E = 0, \quad \nabla \cdot E = \frac{\rho}{\varepsilon_0}
\end{equation*}

Bajo ciertas condiciones geométricas, $\nabla \times E = 0$ implica que existe un potencial electroestático, $U$, esto es, $E = - \nabla U$, verificando que es solución de la ecuación de Poisson $-\Delta U = \frac{\rho}{\varepsilon_0}$ y nuestro modelo se reduce a la ecuación de Vlasov-Poisson:
\begin{align}
 \frac{\partial f}{\partial t} + v \nabla_x f + \frac{q}{m}\nabla_x U \cdot \nabla_v f = 0\\
 \rho(x,t) = q \int f(x,v,t) dv \nonumber\\
 -\Delta U = \frac{\rho}{\varepsilon_0} \nonumber
\end{align}


La \textit{ecuación de Vlasov-Poisson} modela el comportamiento de partículas sometidas a un campo eléctrico sin colisiones:

\begin{align}
\label{eqn:v-poisson}
 \frac{\partial f}{\partial t} + v \nabla_x f + F(x,t) \cdot \nabla_v f = 0\\
 F(x,t) = - \nabla_x U(x,t) \nonumber \\ 
 U(x,t) = \gamma \int_{\mathbb{R}^3} \frac{\sigma(y,t)}{|x-y|} dy \nonumber \\
 \sigma(x,t) = \int_{\mathbb{R}^3} f(x,v,t) dv \nonumber
\end{align}

donde $\gamma = -1$ en el caso gravitacional y $\gamma = 1$ en el caso Coulombiano.

El problema de Cauchy consiste en resolver \eqref{eqn:v-poisson} con condición inicial $f(x,v,0) = f_0(x,v)$.

\section{Vlasov-Poisson con $f_0$ esféricamente simétrica}
Partimos de dos resultados claves: la definición de simetría esférica, y el hecho de que si la condición inicial para la ecuación de Vlasov-Poisson es esféricamente simétrica la solución de \eqref{eqn:v-poisson} también lo es, resultado demostrado en \cite{Batt}.

\begin{definition}
 Una función $f_0:\mathbb{R}_3 \times \mathbb{R}_3 \rightarrow \mathbb{R}$ se dice esféricamente simétrica si y solo sii: 
 \[
 f_0(x,v) = f_0(Ax,Av)
 \] 
 donde $(x,v) \in \mathbb{R}^3\times \mathbb{R}^3$ arbitrario y $A\in O(n)$ es una isometría arbitraria.
\end{definition}

\begin{fact}
 $f_0:\mathbb{R}_3 \times \mathbb{R}_3  \rightarrow \mathbb{R}$ es esféricamente simétrica si y solo podemos encontrar $\phi_0:[0,+\infty[ \times [0,+\infty[ \times [0, \pi]  \rightarrow \mathbb{R}$ verificando $f_0(x,v) = \phi_0(|x|, |v|, \sphericalangle(x,v))$.
 \label{char:sph-symm}
\end{fact}

\begin{proof}

 Si $f_0(x,v) = \phi(|x|,|v|,\sphericalangle(x,v))$ entonces dada $A\in O(n)$ isometría tenemos $f_0(Ax, Av) = \phi_0(|Ax|, |Av|, \sphericalangle(Ax, Av)) = \phi_0(|x|, |v|, \sphericalangle(x,v))$, puesto que las isometrías conservan la norma y los ángulos.
 
 Sean $x,v,x',v'\in \mathbb{R}^3$ con $|x| = |x'|$, $|v| = |v'|$ y verificándose $\sphericalangle(x,v) = \sphericalangle(x',v')$ y sea $A$ la isometría verificando $Ax = x'$ y $Ay = y'$. Entonces $f_0(x,v) = f_0(x',v')$.
 
 Por tanto podemos definir $\phi_0(x,u,\alpha) = f_0(x,v)$ donde $x,v\in\mathbb{R}^3$ arbitrarios tales que $\sphericalangle(x,v) = \alpha, |x|=r, |v|=u$.
\end{proof}

\begin{fact}
Dada $f_0(x,v) = \phi_0(|x|, |v|, \sphericalangle(x,v))$ donde $\phi_0 \in C_0^1(]0,+\infty[^2 \times ]0,+\pi[)$, con $\phi_0\ge 0$. Entonces la solución de Vlasov-Poisson con condición inicial $f_0$ es esféricamente simétrica:
\begin{equation*}
f(x,v,t) = f(Ax, Av, t) \quad \forall x,v\in \mathbb{R}^3 \quad \forall A \in O(3)
\end{equation*}
\end{fact}

Notaremos a partir de ahora $r = |x|, u = |v|, \alpha = \sphericalangle(x,v) = cos^{-1}\left(\frac{x\cdot v}{|x||v|}\right)$. 

Usando la anterior notación y \cref{char:sph-symm}, la solución del problema de Cauchy, $f$, depende únicamente de $r, u, \alpha, t$. Por tanto existe $\phi \in C^1(]0,+\infty[^2\times ]0,\pi[ \times [0,+\infty[)$ verificando:
\begin{equation}
\label{eqn:f=phi}
f(x,v,t) = \phi(r,u,\alpha, t) \quad \forall x,v,t
\end{equation}

A continuación damos un resultado básico de teoría de EDPs.
\begin{lemma}
 Sea $u$ una función armónica en el disco $B(0,r)$. Entonces:
 \[
  \frac{1}{|C(0,r)|}\int_{C(0,r)} u(x) dS_x = u(0)
 \]
 \label{lemma:median-h}
\end{lemma}

\begin{fact}
Sea $\phi_0 \in C_0^1(]0,+\infty[^2 \times ]0,+\pi[)$ con $\phi_0 \ge 0$ y $f_0 = \phi_0$ condición inicial de la ecuación de Vlasov-Poisson. Entonces se cumple, para las cantidades definidas en \eqref{eqn:v-poisson}:
\begin{align*}
 & \sigma(x,t) = \sigma(|x|,t) := \rho(r,t)\\
 & \rho(r, t) = 2\pi \int_0^{\infty} \left(\int_0^\pi \phi(r, u, \alpha, t) sen(\alpha) d\alpha \right)u^2 du\\
 & U(x,t) = 4\pi\gamma \int_0^{+\infty} \frac{\rho(s,t)}{\max(r,s)} s^2 ds\\
 & F(x,t) = \gamma M(r,t) r^{-3} x\\
\end{align*}
donde definimos:
\begin{align*}
 M(r,t) &= \int_{\{x\in \mathbb{R}^3: |x|<r\}} \sigma(x,t) dx = 4\pi \int_0^r\rho(s,t) s^2 ds\\
 M &= \int_{\mathbb{R}^3} \sigma(x,t) dx = 4\pi \int_0^{+\infty}\rho(s,t) s^2 ds\\
\end{align*}
\end{fact}
\begin{proof}
Fijamos $A \in O(3)$. Entonces, usando que $A$ es una matriz ortogonal:
 \[
 \sigma(Ax,t) = \int_{\mathbb{R}^3} f(Ax,v,t) dv \int_{\mathbb{R}^3} f(Ax,Av,t) dv = \int_{\mathbb{R}^3} f(x,v,t) dv = \sigma(x,t)
 \]
 
 Esto prueba que $\sigma(r,t) = \sigma(x,t)$ para todo $x\in \mathbb{R}^3, t\ge 0$.
 
 Tomando ahora $\mathcal{J}:[0,+\infty[ \times [0,\pi[ \times [0,2\pi[ \rightarrow \mathbb{R}^3$, verificando:
 \[\mathcal{J}(r,\alpha, \varphi) = (r\cdot sen(\alpha) \cdot cos(\varphi), r\cdot sen(\alpha) \cdot sen(\varphi), r\cdot cos(\alpha))\]
 tendríamos $det(D\mathcal{J}) = u^2 \cdot sent(\alpha)$. Por teorema de cambio de variable: 
 \begin{align*}
  \sigma(|x|,t) = \sigma(x,t) &= \int_{\mathbb{R}^3} f(x,v,t) dv = \int_{\mathbb{R}^3} \phi(r,u,\alpha,t) dv \\
  &= \int_0^{+\infty} \int_0^\pi \int_0^{2\pi} \phi(r,u,\alpha,t) u^2 \cdot sen(\alpha) d\varphi d\alpha du \\
  &= 2\pi \int_0^{+\infty} \int_0^\pi \phi(r,u,\alpha,t) u^2 \cdot sen(\alpha) d\alpha du
 \end{align*}
 
 Fijamos $x\in \mathbb{R}^3$, y llamamos $r=|x|$.
 \begin{align*}
 U(x,t) &= \gamma \int_{\mathbb{R}^3} \frac{\sigma(y,t)}{|x-y|} dy = \int_0^r \Bigg(\int_{C(0,s)} \frac{\sigma(y,t)}{|x-y|} dS_y\Bigg)ds + 
 \int_r^{+\infty} \Bigg(\int_{C(0,s)} \frac{\sigma(y,t)}{|x-y|} dS_y\Bigg)ds \\
 &= \int_0^r \frac{\sigma(s,t)\cdot |C(0,s)|}{|C(0,s)|} \Bigg(\int_{C(0,s)} \frac{dS_y}{|x-y|}\Bigg) ds + \int_r^{+\infty} \sigma(s,t) \Bigg(\int_{C(0,s)} \frac{dS_y}{|x-y|}\Bigg) ds \\
 &\underset{\textrm{lema \ref{lemma:median-h}}}{=} \int_0^r \frac{4\pi s^2 \sigma(s,t)}{r} ds + 
 \int_r^{+\infty} \sigma(s,t) \Bigg(\int_{C(0,s)} \frac{dS_y}{|x-y|} \Bigg) ds \\
 &= \int_0^r \frac{4\pi s^2 \sigma(s,t)}{r} ds + 
 \int_r^{+\infty} \frac{4\pi s^2 \sigma(s,t)}{s} ds = \int_0^{+\infty} \frac{4\pi s^2 \sigma(s,t)}{\max(r,s)} ds
 \end{align*}
 
 $F(x,t)$ se puede deducir de fácilmente a partir de $\nabla_x U(x,t)$.
\end{proof}

De \eqref{eqn:v-poisson} y de \eqref{eqn:f=phi} deducimos:
\[
\partial_t \phi + v \nabla_x \phi + F(x,t) \cdot \nabla_v \phi = 0\\
\]

donde usando:
\begin{align*}
 v \nabla_x \phi(r, u, \alpha, t) &= v \left(\partial_r \phi \cdot \frac{\partial r}{\partial x} + \partial_\alpha \phi \cdot \frac{\partial \alpha}{\partial x} \right)\\
 &= v\left(\partial_r \phi \cdot \frac{x}{|x|} + \partial_\alpha \phi \cdot \frac{-1}{\sqrt{1 - \left(\frac{xv}{|x||v|}\right)^2}} \cdot \frac{1}{|v||x|^2} \cdot\left(v|x| - xv \frac{x}{|x|}\right)\right)\\
 &= \frac{xv}{|x||v|} u \cdot\partial_r \phi - \partial_\alpha \phi \cdot \frac{1}{\sqrt{1 - \left(\frac{xv}{|x||v|}\right)^2}} \cdot \frac{|v|^2|x|^2 - (xv)^2}{|v|^2|x|^2} \frac{|v|}{|x|}\\
 &\underset{\left\{\begin{array}{c} \frac{xv}{|x||v|} = cos(\alpha)\\ \alpha \in ]0,\infty[\end{array}\right.}{=} \partial_r \phi \cdot cos(\alpha) u - \partial_\alpha \phi \cdot sen(\alpha) \frac{u}{r}
\end{align*}

y  por otro lado:
\begin{align*}
F(x,t) \nabla_v \phi(r, u, \alpha, t) &= F(x,t) \left(\partial_u \phi \cdot \frac{\partial u}{\partial v} + \partial_\alpha \phi \cdot \frac{\partial \alpha}{\partial v} \right) \\
&= F(x,t) \left(\partial_u \phi \cdot \frac{v}{|v|} - \partial_\alpha \phi \cdot \frac{1}{\sqrt{1 - \left(\frac{xv}{|x||v|}\right)^2}} \cdot \frac{1}{|x||v|^2} \cdot\left(x|v| - xv \frac{v}{|v|}\right)\right) \\
&= |x| \gamma M(r,t) r^{-3} x \cdot \partial_u \phi \frac{v}{|v||x|}\\
&- \gamma M(r,t) r^{-3} x \cdot \partial_\alpha \phi \cdot \frac{1}{\sqrt{1 - \left(\frac{xv}{|x||v|}\right)^2}} \cdot \frac{1}{|x||v|^2} \cdot\left(x|v| - xv \frac{v}{|v|}\right)\\
&= \partial_u \phi \cdot \gamma M(r,t) r^{-2} cos(\alpha) - \partial_\alpha \phi \cdot \gamma M(r,t) r^{-3} \cdot sen(\alpha) \frac{r}{u} \\
&= \partial_u \phi \cdot \gamma M(r,t) r^{-2} cos(\alpha) - \partial_\alpha \phi \cdot \gamma M(r,t) r^{-2} u^{-1} \cdot sen(\alpha)
\end{align*}

Luego:

\begin{equation}
\label{eqn:sph-vlasov}
\partial_t \phi + \underbrace{
    \left(\begin{array}{c}
        cos(\alpha)u\\
        \gamma M(r,t) r^{-2} cos(\alpha)\\
        (-\gamma M(r,t) r^{-2} u^{-1} - ur^{-1})\cdot sen(\alpha)
    \end{array}\right)}_{\mathcal{A}(r,u,\alpha,t)}
    \cdot \nabla_{(r,u,\alpha)} \phi = 0
\end{equation}

Sabemos que \eqref{eqn:sph-vlasov} tiene como curvas características $X(t; 0, (r,u,\alpha)) = 
\left(\begin{array}{c} 
       z(t; 0, (r, u, \alpha))\\ q(t; 0, (r, u, \alpha))\\ \theta(t; 0, (r, u, \alpha))
      \end{array}\right)$
verificando:

\begin{equation}
\label{eqn:sph-cc}
\partial_t X(t; 0,(r,u,\alpha)) = \mathcal{A}\big(X(t;0(r,u,\alpha)),t\big) = \mathcal{A}\big(z,q,\theta,t\big)
\end{equation}

\begin{fact} Se cumple:
 $zq \cdot sen(\theta)$ es independiente de $t$ y además $z$ es solución de la ecuación diferencial:
 \begin{align}
 z_{tt} - (ru \cdot sen\alpha)^2 \cdot z^{-3} - \gamma M(z,t) z^{-2} = 0\nonumber\\
 z(r,u,\alpha, 0) = r \nonumber\\
 z_t(r,u,\alpha,0) = u\cdot cos(\alpha)
 \label{eqn:curves}
 \end{align}
\end{fact}
\begin{proof}
 \begin{align*}
  \partial_t(zq \cdot sen(\theta)) &= z_t q\cdot sen(\theta) + zq_t \cdot sen(\theta) + zq \cdot cos(\theta) \cdot \theta_t \\
  &= cos(\theta) q^2 \cdot sen(\theta) + z\gamma M(z,t) z^{-2} \cdot cos(\theta) \cdot sen(\theta) \\
  &- zq \cdot cos(\theta) \Bigg(\gamma M(z,t) z^{-2} q^{-1} \cdot sen(\theta)  + qz^{-1}\cdot sen(\theta) \Bigg) = 0
 \end{align*}

Por otro lado, derivando la primera componente de $X(t)$, y teniendo en cuenta que $\left(\begin{array}{c} z(0)\\ q(0)\\ \theta(0) \end{array}\right) = 
\left(\begin{array}{c} r\\ u\\ \alpha\end{array}\right)$ y por tanto $zq\cdot sen(\theta) = ru\cdot sen(\alpha)$:
\begin{equation}
 z_{tt} - (ru \cdot sen(\alpha))^2 \cdot z^{-3} - \gamma M(z,t) z^{-2} = 0
\end{equation}
\end{proof}

\begin{lemma}
Fijado $t\ge 0$, $\mathcal{J}_t:(x,v) \mapsto \bigg(X(t;0,x,v), V(t;0,x,v)\bigg)$ es un cambio de variable con jacobiano idénticamente $1$ para el problema de Vlassov-Poisson \eqref{eqn:v-poisson}.
\label{lemma:cv-vp}
\end{lemma}
\begin{proof}
Por simplicidad de notación, llamaremos:
\begin{align*}
(X_1, X_2, X_3) &:= X\\
(X_4, X_5, X_6) &:= V\\
\mathcal{J}(t,x,v) &:= \mathcal{J}_t(x,v)
\end{align*}

Sabemos que $\mathcal{J}_t(x,v)$ es una biyección de $\mathbb{R}^3 \times \mathbb{R}^3$ en sí mismo, por la teoría clásica de transporte, y con la notación antes descrita:
 \[
 det(D\mathcal{J}(t,x,v)) := det\bigg(\frac{\partial{X_i}}{\partial x_j}(t;0,x,v)\bigg)_{i,j}
 \]

 Por un teorema de clase sabemos:
 \[
 \frac{\partial det(D\mathcal{J}(t,x,v))}{\partial t} = det(D\mathcal{J}(t,x,v)) \cdot \underbrace{div_{x,v}((x,v) \mapsto (v,F(x,t)))}_0 = 0
 \]
 Luego $det(D\mathcal{J}(t,x,v)) = det(D\mathcal{J}(0,x,v)) = 1$.
\end{proof}

\begin{fact}
Llamando $z = z(t;0,r,u,\alpha), q = q(t;0,r,u,\alpha), \theta = \theta(t;0,r,u,\alpha)$, tenemos que para todo $\varphi \in L^{\infty}(]0,+\infty[^2 \times ]0,\pi[)$:
\[
    \int_{\mathbb{R}^3}\int_{\mathbb{R}^3} \varphi(r,u,\alpha) \cdot \phi(r,u,\alpha,t) dv dx = 
    \int_{\mathbb{R}^3}\int_{\mathbb{R}^3} \varphi(z,q,\theta) \cdot \phi_0(r,u,\alpha,t) dv dx
\]

En particular:
\[
M = 4\pi \int_0^{\infty} \rho(s,t) s^2 ds = \int_{\mathbb{R}^3} \int_{\mathbb{R}^3} \phi(r,u,\alpha,t) dv dx = \int_{\mathbb{R}^3}\int_{\mathbb{R}^3} \phi_0(r,u,\alpha,t) dv dx
\]
\end{fact}

\begin{proof}
Sean $X(t; 0, x_0, v_0), V(t; 0, x_0, v_0)$ las curvas características del problema \eqref{eqn:v-poisson} con $f_0$ esféricamente simétrica.

 \begin{align*}
 &\int_{\mathbb{R}^3}\int_{\mathbb{R}^3} \varphi(r,u,\alpha) \cdot \phi(r,u,\alpha,t) dv dx = \int_{\mathbb{R}^3}\int_{\mathbb{R}^3} \varphi(|x|,|v|,\sphericalangle(x,v)) \cdot f(x,v,t) dv dx \\
 &= \int_{\mathbb{R}^3}\int_{\mathbb{R}^3} \varphi(|x|,|v|,\sphericalangle(x,v)) \cdot f_0(X(0;t,x,v)) dv dx \underset{\textrm{Haciendo el cambio } x,v \rightarrow \mathcal{J}_t(x,v)}{=}\\
 &=\int_{\mathbb{R}^3}\int_{\mathbb{R}^3} \varphi\bigg(|X(t;0,x,v)|,|V(t;0,x,v)|,\sphericalangle\bigg(X(t;0,x,v), V(t;0,x,v)\bigg)\bigg) \cdot f_0(x,v) dv dx \\
 &= \int_{\mathbb{R}^3}\int_{\mathbb{R}^3} \varphi(z,q,\theta) \cdot \phi_0(r,u,\alpha) dv dx
 \end{align*}

 donde se ha usado el lema \ref{lemma:cv-vp} en el cambio de variable y $X\bigg(0;t,X(t,0,x,v)\bigg) = (x,v)$
\end{proof}


Llamando $\mathcal{C} \subseteq ]0, +\infty[^2 \times ]0, \pi[$ al soporte de $\phi_0$, puede verse fácilmente que existen $r_0, R_0, l_0, L_0, U_0, \rho_0 > 0$, dependientes sólo de $\phi_0$, verificando que para todo $(r,u,\alpha,t) \in \mathcal{C} \times ]0, +\infty[$ se cumple:
\begin{align}
 r_0 \le z(r,u,\alpha,t) \le R_0 + U_0 t \nonumber\\
 l_o \le ru \le L_0 \nonumber\\
 l_0 \le ru \cdot sen(\alpha) = zq \cdot sen(\theta) \le L_0 \nonumber\\
 q(r,u,\alpha, t) \le U_0 \nonumber \\
 \rho(r,t) \le \rho_0 \label{eqn:zbounds}
\end{align}

donde en la última integral tenemos que $z,q,\theta$ son las curvas características de \eqref{eqn:sph-vlasov} verificando 
\[
    \left(\begin{array}{c} z(0)\\ q(0)\\ \theta(0) \end{array}\right) = 
    \left(\begin{array}{c} r_0\\ u_0\\ \alpha_0 =\sphericalangle(x_0, v_0) \end{array}\right)
\]

Usaremos una desigualdad fundamental para demostrar la última proposición de esta introducción: la desigualdad de Gronwall.

\begin{lemma}[Desigualdad de Gronwall]
Sean $\alpha, \beta \ge 0, u$ funciones continuas, con $\alpha$ no decreciente, verificando para todo $t\in I$, siendo $I$ es un intervalo de tipo $[a,b], [a,b[$ o $[a,+\infty[$, se cumple:
\[
 u(t) \le \alpha(t) + \int_a^t \beta(s) u(s)
\]

Entonces: $u(t) \le \alpha(t) \cdot exp\left\{\int_a^t \beta(s) ds\right\}$ para todo $t\in I$.
\label{ineq:gronwall}
\end{lemma}

\begin{fact}
 Existe una función creciente $K(t) \ge 1$ dependiente sólo de $\phi_0$ verificando que para todo $c_1=(r_1,u_1,\alpha_1), \, c_2=(r_2,u_2,\alpha_2) \in \mathcal{C}$ y para todo $t\ge 0$:
 \[|z(c_1,t) - z(c_2,t)| + |z_t(c_1,t) - z_t(c_2,t)| \le K(t)|c_1 - c_2|\]
 \label{fact:ineq-curves}
\end{fact}

\begin{proof}
 Fijamos $c_1, c_2$ y notamos:
 \begin{align*}
    z_i(t) &= z(t; 0, c_i)\\
    Q(t) &= \sup\{|z_1(s) - z_2(s)| + |z_1'(s) - z_2'(s)|: 0 \le s \le t\}
 \end{align*}
 
 
 Usando que $sen$ es una función Lipschitziana con constante de Lipschitzianidad $1$:
 \begin{align}
 \bigg|r_1 u_1 \cdot sen(\alpha_1) - r_2 u_2 \cdot sen(\alpha_2)\bigg| &= \nonumber\\
 \bigg|r_1 u_1 (sen(\alpha_1) - sen(\alpha_2) + sen(\alpha_2)) - (r_2 - r_1 + r_1) u_2 \cdot sen(\alpha_2)\bigg| \nonumber \\
 \le r_1 u_1 \bigg|sen(\alpha_1) - sen(\alpha_2)\bigg| + r_1\bigg|u_1 - u_2\bigg| \cdot sen(\alpha_2) + u_2\bigg|r_1 - r_2\bigg| \cdot sen(\alpha_2) \nonumber\\
 \underset{|sen(\alpha_1) - sen(\alpha_2)| \le |\alpha_1 - \alpha_2|}{\le} L_0|\alpha_1 - \alpha_2| + R_0|u_1 - u_2| + U_0|r_1 - r_2| \le K|c_1 - c_2|
 \label{ineq:sins}
 \end{align}
 
 Análogamente:
 \[
  \bigg|r_1 u_1 cos(\alpha_1) - r_2 u_2 cos(\alpha_2)\bigg| \le K|c_1 - c_2|
 \]
 donde s.p.g. podemos suponer idéntica constante $K$.
 
 Además:
 
\[
 |M(z_1, t) - M(z_2, t)| = \bigg|4\pi \int_{z_1}^{z_2} \rho(s,t) s^2 ds \bigg| \le 4\pi(R_0 + U_0t)^2 \rho_0 |z_1 - z_2|
\]

 
 Así, para todo $t\ge 0$ se cumple, usando fuertemente \eqref{eqn:curves}:
 \begin{align*}
  |z_1'' - z_2''| &=
  \bigg|(r_1 u_1 \cdot sen(\alpha_1))^2 z_1^{-3} - (r_2 u_2 \cdot sen(\alpha_2))^2 z_2^{-3} + \gamma M(z_1, t) z_1^{-2} - \gamma M(z_2, t) z_2^{-2} \bigg| \\
  &= \bigg|(r_1 u_1 \cdot sen(\alpha_1))^2 z_1^{-3} - (r_2 u_2 \cdot sen(\alpha_2))^2 (z_2^{-3} - z_1^{-3} + z_1^{-3}) \\
  &+ \gamma M(z_1, t) z_1^{-2} - \gamma M(z_2, t) (z_2^{-2} - z_1^{-2} + z_1^{2}) \bigg| \\
  &\le \bigg|(r_1 u_1 \cdot sen(\alpha_1)^2 - (r_2 u_2 \cdot sen(\alpha_2)^2 \bigg| z_1^{-3} + (r_2 u_2 \cdot sen(\alpha_2))^2 \bigg|z_1^{-3} - z_2^{-3}\bigg| \\
  &+ \bigg|M(z_1,t) - M(z_2,t) \bigg|z_1^{-2} + M(z_2, t) \bigg|z_1^{-2} - z_2^{-2}\bigg| \\
  &\underset{\textrm{\eqref{eqn:zbounds}}}{\le} \bigg|r_1 u_1 \cdot sen(\alpha_1)+ r_2 u_2 \cdot sen(\alpha_2))\bigg|\bigg|r_1 u_1 \cdot sen(\alpha_1) - r_2 u_2 \cdot sen(\alpha_2)\bigg| r_0^{-3} \\
  &+ L_0^2 z_1^{-3}z_2^{-3} \bigg|z_2^3 - z_1^3 \bigg| + \bigg|M(z_1,t) - M(z_2,t)\bigg| r_0^{-2} + Mz_1^{-2} z_2^{-2} \bigg|z_2^2 - z_1^2\bigg| \\
  &\underset{\label{eqn:zbounds}}{\le} 2L_0 r_0^{-3}\bigg|r_1 u_1 \cdot sen(\alpha_1) - r_2 u_2 \cdot sen(\alpha_2)\bigg|
  + L_0^{2} r_0^{-6}\bigg|z_1^2 + z_1 z_2 + z_2^2\bigg|\bigg|z_1 - z_2\bigg|\\
  &+ r_0^{-2}\bigg|M(z_1,t) - M(z_2,t)\bigg| + Mr_0^{-4}\bigg|z_1 + z_2\bigg| \bigg|z_1 - z_2\bigg| \\
  &\le 2L_0 r_0^{-3} K|c_1 - c_2| 
  + L_0 r_0^{-6} 3(R_0 + U_0t)^2 |z_1 - z_2| \\
  &+ r_0^{-2} 4\pi (R_0 + U_0t)^2 \rho_0 |z_1 - z_2| + Mr_0^{-4} 2(R_0 + U_0t) |z_1 - z_2|\\
  &= K|c_1 - c_2| + K(t)|z_1 - z_2|
 \end{align*}
 
 donde $K(t)$ es una función creciente y se ha renombrado la constante $K$.
 
 Se desprende:
 \begin{align*}
  |z_1 - z_2| + |z_1' - z_2'| = \left|r_1 - r_2 + \int_0^t (z_1' - z_2')\right| + \left|u_1 \cdot cos(\alpha_1) - u_2 \cdot cos(\alpha_2) + \int_0^t (z_1'' - z_2'') \right|\\
  \le |r_1 - r_2| + |u_1 \cdot cos(\alpha_1) - u_2 \cdot cos(\alpha_2)| + \int_0^t \bigg(|z_1' - z_2'| + |z_1'' - z_2''|\bigg) \\
  \le |r_1 - r_2| + |u_1 \cdot cos(\alpha_1) - u_2 \cdot cos(\alpha_2)| + \int_0^t \bigg(Q(s) + K|c_1 - c_2| + K(s)Q(s)\bigg) ds \\
  \le |r_1 - r_2| + |u_1 \cdot cos(\alpha_1) - u_2 \cdot cos(\alpha_2)| + Kt|c_1 - c_2| + (1 + K(t)) \int_0^t Q(s) ds\\
  \le \widetilde{K}(t)|c_1 - c_2| + \widetilde{K}(t) \int_0^t Q(s) ds
 \end{align*}
 para cierta $\widetilde{K} \ge 1$ creciente. 
 
 Al ser $\widetilde{K}(t)$ es creciente, para todo $0 \le s \le t$:
 \[
    |z_1(s) - z_2(s)| + |z_1'(s) - z_2'(s)|  \le \widetilde{K}(t)|c_1 - c_2| + \widetilde{K}(t) \int_0^t Q(s)
 \]

 Luego para $t\in [0,T]$:
 \[
  Q(t) \le \widetilde{K}(t)|c_1 - c_2| + \widetilde{K}(t) \int_0^t Q(s) \le \widetilde{K}(T)|c_1 - c_2| + \widetilde{K}(T) \int_0^t Q(s)ds
 \]
 y por la desigualdad de Gronwall \ref{ineq:gronwall}:
 \[
  Q(t) \le \widetilde{K}(T)|c_1 - c_2| \cdot e^{C(T)t}
 \]
 
 Por consiguiente para $T$ arbitrario:
 \[
  Q(T) \le \widetilde{K}(T)\cdot e^{TC(T)} \cdot |c_1 - c_2|
 \]
 
\end{proof}

\newpage
\section{Modelo discreto}
Asumamos que $C$, soporte de $\phi_0$, puede escribirse como $C= \bigcup_{i=1}^N S_i$ donde $S_i$ son conexos, y que:
\[
    \delta = \max_{i=1,\ldots, N} diam(S_i) \le \{1,r_0\}
\]

Fijamos $c_i = (r_i, u_i, \alpha_i) \in S_i$ y llamamos:
\begin{itemize}
\item $L_i = r_i u_i \cdot sen(\alpha_i)$
\item $M_i = \int_{S_i} \phi_0(r,u,\alpha) dv dx$
\item $z_i(t) = z(t;0,c_i)$
\item $\xi(r) = \min\left(\frac{r}{\delta},1\right) \mathds{1}_{]0,\infty[}$
\end{itemize}

$z_i$ cumpliría la ecuación \eqref{eqn:curves}, luego:
\begin{align*}
 z_i'' - L_i^2 z_i^{-3} - \gamma M(z_i, t) z_i^{-2} = 0\\
 z_i(0) = r_i\\
 z_i'(0) = u_i \cdot cos(\alpha_i)
\end{align*}

Tomaremos las siguientes ecuaciones diferenciales, que intuitivamente deberían aproximar $z_i$ para $i=1, \ldots, N$:
\begin{align}
 \label{eqn:approx-curves}
 \widetilde{z}_i'' - L_i^{2} \widetilde{z}_i^{-3} - \gamma \widetilde{M}(\widetilde{z}_i, t) \widetilde{z}_i^{-2} = 0 \nonumber\\
 \widetilde{z}_i(0) = r_i \nonumber\\
 \widetilde{z}'_i(0) = u_i \cdot cos(\alpha_i)
\end{align}
donde $\widetilde{M}(r,t) = \sum_{i=1}^N M_i \xi(r-\widetilde{z}_i(t))$, que se puede pensar como una suma ponderada de las masas de los conexos cuya $z_i \le r$.

Con estas condiciones, \eqref{eqn:approx-curves} y como $r_0 \le r_i \le R_0$, puede tomarse la solución maximal (definida en $[0,T[$), verificando:
\begin{equation}
    \frac{1}{2}r_0 \le \widetilde{z}_i(t) \le 2(R_0 + U_0t)
\label{eqn:barzbounds}
\end{equation}

Además definiremos $\dtilde{M}(r,t) = \sum_{i=1}^N M_i \xi(r-z_i(t))$.
\begin{theorem}
 Existe una función $\widetilde{K}$ (dependiente sólo de $\phi_0$) verificando que para todo $i = 1, \ldots, N$, para todo $t\in [0,T[$:
 \begin{align*}
  |z(c,t) - \widetilde{z}_i(t)| + |z'(c,t) - \widetilde{z}'_i(t)| &\le \ktilde(t) \delta \qquad \forall c= (r,u,\alpha) \in S_i \\
  |M(s,t) - \widetilde{M}(s,t)| &\le \ktilde(t)\delta \qquad \forall s \ge 0
 \end{align*}
 Además $T \rightarrow + \infty$ cuando $\delta \rightarrow 0$.
 \label{th:fundamental}
\end{theorem}

\begin{proof}
 Fijamos $t \ge 0, c'\in C$ y llamamos $z = z(c',t)$. Definimos:
 \begin{align*}
 J_1 &= \{j: z_j(t) \le z\}\\
 J_2 &= \{j: z_j(t) \le z - K(t)\delta\}
 \end{align*}
 
 Si $c\in S_j$ y $j\in J_1$, por \cref{fact:ineq-curves} se deduce:
 \[
    z(c,t) \le z(c_j,t)| + K(t)|c - c_j| = z_j(t) + K(t) |c - c_j| \le z_j(t) + K(t)\delta \le z + K(t) \delta
 \]
 
 Por tanto:
 \[
    \bigcup_{j\in J_1} S_j \subseteq \{c \in \mathbb{R}^3: z(c,t) \le z + K(t) \delta\} = \overline{A}
 \]
 
 
 \begin{align*}
 \dtilde{M}(z,t) &= \sum_{j = 1}^M M_j \xi(z - z_j(t)) \underset{z <      
 z_j(t) \Rightarrow \xi(z - z_j(t)) = 0}{\le} \sum_{j \in J_1} M_j\\
  &= \int_{\bigcup_{j\in J_1} S_j} \phi_0(r,u,\alpha) dv dx \le \int_{\overline{A}} \phi_0(r,u,\alpha) dv dx\\
  &= \int_{\{(x,v):r \le z + K(t)\delta\}} \phi(r,u,\alpha,t) dv dx = M(z + K(t)\delta, t)\\
  & = M(z,t) + 4\pi \int_{z}^{z+K(t)\delta} s^2 \rho(s,t) ds \\ 
  &\le M(z,t) + 4\pi \rho_0 (R_0 + U_0t + K(t) \delta)^2 K(t) \delta = M(z, t) + \ktilde_1(t) \delta
 \end{align*}

 Por otro lado, si $c \in S_j$ y $z(c,t) \le z - 2K(t) \delta$ entonces por \cref{fact:ineq-curves}, $z_j(t) \le z(c,t) + K(t) \delta \le z - K(t) \delta$. Por tanto:
 \[
  \underline{A} = C \cap \{c\in \mathbb{R}^3:z(c,t) \le z - 2K(t) \delta\} \subseteq \bigcup_{j\in J_2} S_j
 \]
 
 Por tanto:
 \begin{align*}
 \dtilde{M}(z,t) &= \sum_{j = 1}^M M_j \xi(z - z_j(t)) \underset{z_j \le z - K(t)\delta \implies z - z_j(t) \ge \delta \Rightarrow \xi(z - z_j(t)) = 1}{\ge} \sum_{j \in J_2} M_j\\
  &= \int_{\bigcup_{j\in J_2} S_j} \phi_0(r,u,\alpha) dv dx \ge \int_{\underline{A}} \phi_0(r,u,\alpha) dv dx\\
  &= \int_{\{(x,v):r \le z - 2K(t)\delta\}} \phi(r,u,\alpha,t) dv dx = M(\max(0,z-2K(t)\delta), t)\\
  & = M(z,t) - 4\pi \int_{\max(0,z-2K(t)\delta)}^z  s^2 \rho(s,t) ds \\ 
  &\ge M(z,t) - 4\pi \rho_0 R_0^2 2K(t)\delta \ge M(z, t) - \ktilde_2(t) \delta
 \end{align*}
 
 Tomando $\ktilde_3 = \max(\ktilde_1, \ktilde_2)$ hemos probado que:
 \begin{equation}
 |M(r,t) - \dtilde{M}(r,t)| \le \ktilde_3(t) \delta
 \label{eqn:m-mdtilde1}
 \end{equation}
 para $r \in z(C,t)$, pero queremos el resultado para todo $r\ge 0$.
 
 Observamos que si $r \ge R_0 + U_0 t + \delta$ entonces $M(r,t) = M = \dtilde{M}(r,t)$, y si $0\le r \le r_0$ entonces $M(r,t) = \dtilde{M}(r,t) = 0$. Así, nos restringimos al caso $r_0 \le r \le R_0 + U_0t + 1$. Supongamos que existe $r_1 \in ]r_0, R_0 + U_0t + 1[ \cap z(C, t)^c$ (caso opuesto ya habríamos acabdo la demostración).
 
 Además, $C$ es compacto y $z$ continua, luego podemos considerar s.p.g. $z(C,t) = [r_0, r_2]$, donde $]r_2,r_1[ \cap z(C,t) = \emptyset$.
 Deducimos:
 \begin{align}
  M(r_1, t) - M(r_2, t) &= 4\pi\int_{r_2}^{r_1} s^2 \rho(s,t) ds = \int_{\{(x,v): r\in ]r_2, r_1[\}} \phi(r,u,\alpha,t) dv dx \nonumber\\
  &= \int_{\{c\in\mathbb{R}^3: z(c,t) \in ]r_2, r_1[\}} \phi_0(c) dv dx
  \underset{\{c: z(c,t) \in ]r_2,r_1[\} \cap C = \emptyset}{=} 0
  \label{eqn:mdiff}
 \end{align}
 
 Si $z_j(t) = z_j(c_j,t) \in ]r_2 - \delta, r_2]$ y dado $c'\in S_j$, por \cref{fact:ineq-curves} deducimos:
 \[
 -K(t)\delta -\delta \le z(c',t) - z_j(t) - \delta \le z(c',t) - (r_2 - \delta) - \delta = z(c',t) - r_2 \le z(c',t) - z_j(t) \le K(t)\delta
 \]
 
 Llamamos:
 \[
 J_3 = \{j: z_j(t) \in ]r_2 - \delta, r_2]\}
 \]
 
 De $z_j(t) \notin ]r_2,r_1[$ y usando $r_1 > r_2$ deducimos que si $z_j(t) \le r_2 - \delta$ entonces $\xi(r_1 - z_j) = \xi(r_2 - z_j) = 1$; si $z_j(t) > r_1$ entonces $\xi(r_1 - z_j) = \xi(r_2 - z_j) = 0$ y si $z_j \in ]r_2 - \delta, r_2[$ entonces $\xi(r_1 - z_j) > \xi(r_2 - z_j)$. Juntando toda esta información:
 \begin{align}
 &0 \le \dtilde{M}(r_1,t) - \dtilde{M}(r_2,t) = \sum_{j=1}^N M_j \bigg\{\xi(r_1 - z_j(t)) - \xi(r_2 - z_j(t))\bigg\} \le \sum_{j \in J_3} M_j \nonumber\\ 
 &=\int_{\cup_{j\in J_3} S_j} \phi_0(r,u,\alpha) dv dx \le \int_{\{c : |z(c,t) - r_2| \le \delta + K(t) \delta\}} \phi_0(c) \nonumber\\
 &= \int_{\{(x,v): ||x| - r_2| \le \delta + K(t) \delta\}} \phi(c,t) dv dx = 4\pi \int_{max(0, r_2 - \delta - K(t)\delta}^{r_2 + \delta + K(t) \delta} s^2 \rho(s,t) ds \nonumber\\
 &\le 4\pi (r_2 + 1 + K(t))^2 \rho_0 2(1+ K(t))\delta \nonumber\\
 &\le 4\pi (R_0 + U_0t + 2 + K(t))^2 \rho_0 2 (1 + K(t))\delta \le \ktilde_4(t)\delta
 \label{eqn:mtildediff}
 \end{align}

 Como $r_2 \in z(C,t)$, entonces $|M(r_2,t) - \dtilde{M}(r_2,t)| \le \ktilde_3(t) \delta$ por \eqref{eqn:m-mdtilde1}.
 
 En definitiva:
 \begin{align}  
|M(r_1,t) - \dtilde{M}(r_1,t)| &= |M(r_1,t) - M(r_2,t) + M(r_2,t) - \dtilde{M}(r_2,t) + \dtilde{M}(r_2,t) - \dtilde{M}(r_1,t)| \nonumber \\
&\le \underbrace{|M(r_1,t) - M(r_2,t)|}_{\textrm{\eqref{eqn:mdiff}}} + |M(r_2,t) - \dtilde{M}(r_2,t)| + \underbrace{|\dtilde{M}(r_2,t) - \dtilde{M}(r_1,t)|}_{\textrm{\eqref{eqn:mtildediff}}} \nonumber \\
&\le 0 + \ktilde_3(t) \delta + \ktilde_4(t) \delta \label{eqn:m-mdtilde2}
 \end{align}

 Hemos probado, por \eqref{eqn:m-mdtilde1} y  \eqref{eqn:m-mdtilde2}
 \begin{equation}
  |M(r,t) - \dtilde{M}(r,t)| \le \ktilde_5(t) \delta, \qquad r\ge 0, t\ge 0
  \label{eqn:m-mdtilde}
 \end{equation}
 donde $\ktilde_5 = \ktilde_3 + \ktilde_4$

 Sea $0\le t < T$ y definimos $||z-\widetilde{z}||(t) = \sup_{0\le s \le t} \max_{1\le i\le N} |z_i(s) - \widetilde{z}_i(s)|$.
 
 Fijamos $r\ge 0, t\in [0, T[$. Si $z_j(t) \ge r + ||z - \widetilde{z}||(t) \ge r + ||z - \widetilde{z}||(t) - \delta$, entonces:
 \begin{align*}
  r - z_j(t) \le 0 \Rightarrow \xi(r - z_j(t)) = 0\\
  r - \widetilde{z}_j(t) \le r - (z_j(t) - ||z-\widetilde{z}||(t)) \le 0 \Rightarrow \xi(r - \widetilde{z}_j(t)) = 0
 \end{align*}
 
 Análogamente, si $z_j(t) \le r - ||z - \widetilde{z}||(t) - \delta$, entonces:
  \begin{align*}
  \xi(r - z_j(t)) = 1\\
  \xi(r - \widetilde{z}_j(t)) = 1
 \end{align*}
 
 Como $0 \le \xi(r - z_j(t)) \le 1, 0 \le \xi(r - \widetilde{z}_j(t)) \le 1$ entonces $|\xi(r - \widetilde{z}_j(t)) - \xi(r - z_j(t))| \le 1$. Llamaremos $J_4 = \{j: |r-z_j(t)| \le ||z - \widetilde{z}||(t) + \delta\}$.  Si $j\in J_4, c' \in S_j$, entonces, por \cref{fact:ineq-curves}:
 \begin{align*}
  |z(c',t) - r| &= |z(c',t) - z_j(t) + z_j(t) - r| \le |z(c',t) - z_j(t)| + |z_j(t) - r|  \\
  &\le ||z-\widetilde{z}||(t) + \delta + K(t)\delta\\
  \bigcup_{j\in J_4} S_j &\subseteq \{c\in \mathbb{R}^3: |z(c,t) - r| \le ||z-\widetilde{z}||(t) + \delta + K(t)\delta\} = \underline{B}
 \end{align*}

 Se desprende, llamando $\varepsilon(t) = ||z-\widetilde{z}||(t) + \delta + K(t)\delta \underset{\textrm{\eqref{eqn:barzbounds}, \eqref{eqn:zbounds}}}{\le} 3(R_0 + U_0 t) + 1 + K(t)$:
 \begin{align*}
 |\widetilde{M}(r,t) - \dtilde{M}(r,t)| &= \left|\sum_{j=1}^N \xi(r - \widetilde{z}_j(t)) - \xi(r - z_j(t)) \right| \le \sum_{j\in J_4} M_j
 \le \int_{\bigcup_{j\in J_4} S_j} \phi_0(c) dv dx\\
 &\le \int_{\underline{B}} \phi_0(c) dv dx
 = \int_{\{(x,v):||x| - r| < \varepsilon(t)\}} \phi(r,u,\alpha.t) dv dx \\
 &= 4\pi \int_{\max(0,r-\varepsilon(t))}^{r+\varepsilon(t)}  s^2 \rho(s,t) ds \le 4\pi(r + \varepsilon(t))^2 \rho_0 2\varepsilon(t)
 \end{align*}
 
 Si $r > 2(R_0 + U_0t) + 1$ entonces $\widetilde{M}(r,t) = \dtilde{M}(r,t) = M$. Si $r \le 2(R_0 + U_0t) + 1$ entonces:
 \begin{align}
  |\widetilde{M}(r,t) - \dtilde{M}(r,t)| &\le 4\pi\bigg\{5(R_0 + U_0t) + 2 + K(t)\bigg\}^2 \rho_0 2\epsilon(t) \nonumber\\
  &\le \ktilde_6(t)\bigg\{||z-\widetilde{z}||(t) + \delta + K(t)\delta\bigg\} 
  \nonumber \\
  &\le \ktilde_7(t)||z - \widetilde{z}||(t) + \ktilde_7(t)\delta \label{eqn:barbarmbound}
 \end{align}
 
 Deducimos para todo $r\ge 0$ y para todo $t \ge 0$:
 \begin{align}
  |M(r,t) - \widetilde{M}(r,t)| &= |M(r,t) - \dtilde{M}(r,t) + \dtilde{M}(r,t) - \widetilde{M}(r,t)| \nonumber\\
  &\le |M(r,t) - \dtilde{M}(r,t)| + |\dtilde{M}(r,t) - \widetilde{M}(r,t)| \nonumber\\
  &\underset{\textrm{\eqref{eqn:m-mdtilde}, \eqref{eqn:barbarmbound}}}{\le} \ktilde_7(t)||z - \widetilde{z}||(t) + (\ktilde_5(t)+\ktilde_7(t))\delta \nonumber\\
  &\le \ktilde_8(t)||z - \widetilde{z}||(t) + \ktilde_8(t)\delta
  \label{ineq:diffm}
 \end{align}
donde $\ktilde_8 = 2(\ktilde_5 + \ktilde_7)$

Nos queda acotar $||z - \widetilde{z}||(t)$. Fijamos $t \in [0,T[$.

Partimos de:
\begin{align}
 |M(z_i,t) - M(\widetilde{z}_i,t)| = \bigg|\int_{\widetilde{z}_i}^{z_i} 4\pi s^2 \rho(s,t) ds \bigg| \le \ktilde_9(t) ||z-\widetilde{z}||(t) \nonumber\\
 \widetilde{z}_i^{-n} \ge \bigg(\frac{1}{2} r_0\bigg)^{n} \qquad n\in\mathbb{N} \nonumber\\
 \widetilde{z}_i^n \le (2(R_0+U_0t))^n \nonumber\\
 z_i^n \le (R_0+U_0t)^n \nonumber\\
 |z_i - \widetilde{z}_i| \le ||z-\widetilde{z}||(t) \nonumber\\
 |z_i^2 - \widetilde{z}_i^2| = |z_i + \widetilde{z}_i||z_i - \widetilde{z}_i| \nonumber\\
 |z_i^3 - \widetilde{z}_i^3| = |z_i^2 + z_i\widetilde{z}_i + \widetilde{z}_i^2||z_i - \widetilde{z}_i|
 \label{ineq:basics}
\end{align}

y escribiendo:
\begin{align}
 |z_i''(t) - \widetilde{z}_i''(t)| &= \bigg|L_i^2 z_i^{-3} + \gamma M(z_i,t) z_i^{-2} - L_i^2 \widetilde{z}_i^{-3} - \gamma \widetilde{M}(\widetilde{z}_i, t) \widetilde{z}_i^{-2} \bigg| \nonumber\\
 &\le L_i^2 |z_i^{-3} - \widetilde{z}_i^{-3}| + M(z_i, t) |z_i^{-2} - \widetilde{z}_i^{-2}| + \widetilde{z}_i^{-2} |M(z_i,t) - \widetilde{M}(\widetilde{z}_i, t)|\nonumber\\
 &\le L_0 z_i^{-3}\widetilde{z}_i^{-3} |z_i^3 - \widetilde{z}_i^{3}| +
 Mz_i^{-2} \widetilde{z}_i^{-2} |z_i^2 - \widetilde{z}_i^{2}| \nonumber \\
 &+ \widetilde{z}_i^{-2} \bigg\{|M(z_i,t) - M(\widetilde{z}_i,t)| + |M(\widetilde{z}_i,t) - \widetilde{M}(\widetilde{z}_i,t)|\bigg\} \nonumber\\
 &\underset{\textrm{\eqref{ineq:basics}}}{\le} \ktilde_9(t) \delta + \ktilde_9(t)||z - \widetilde{z}||(t)
 \label{ineq:zibound}
\end{align}

Definimos:

\[
Q(t) := \max_{0 \le s \le t}\max_{i = 1, \ldots, N} \bigg\{|z_i(s) - \widetilde{z}_i(s)| + |z_i'(s) - \widetilde{z}'_i(s)| \bigg\} 
\]

Como $z_i(0) = \widetilde{z}_i(0), z_i'(0) = \widetilde{z}_i'(0)$, para todo $0\le v \le t$:

\begin{align*}
 |z_i(v) - \widetilde{z}_i(v)| + |z_i'(v) - \widetilde{z}_i'(v)| &\le \int_0^v (|z_i(s) - \widetilde{z}_i(s)| + |z_i'(s) - \widetilde{z}_i'(s)|) ds\\
 &\le \int_0^t (|z_i(s) - \widetilde{z}_i(s)| + |z_i'(s) - \widetilde{z}_i'(s)|) ds\\
 &\le \int_0^t \bigg\{Q(s) + \ktilde_9(s)\delta + \ktilde_9 Q(s)\bigg\} ds \\
 &\le \ktilde_{10}(t)\delta + \ktilde_{10}(t) \int_0^{t} Q(s) ds
\end{align*}

Para $0\le v\le t$ arbitrario, luego $Q(t) \le \ktilde_{10}(t)\delta + \ktilde_{10}(t) \int_0^{t} Q(s) ds$, donde $0\le t < T$ era arbitrario. Por desigualdad de Gronwall \ref{ineq:gronwall} :
\[
Q(v) \le \ktilde_{10}(t) \delta e^{\ktilde_{10}(t) v} \Rightarrow Q(t) \le \ktilde_{11}(t) \delta
\]

Por tanto, usando lo anterior y \cref{fact:ineq-curves}, si $c\in S_i$ y para todo $0\le t < T$, entonces:
\begin{align}
 |z(c,t) - \widetilde{z}_i(t)| + |z_t(c,t) - \widetilde{z}_i'(t)| &\le 
 |z(c,t) - z_i(t)| + |z_i(t) - \widetilde{z}_i(t)| \nonumber\\
 &+ |z'(c,t) - z_i'(t)| + |z_i'(t) - \widetilde{z}_i'(t)| \nonumber\\
 &\le K(t)\delta +\widetilde{K}_{11}(t)\delta \le \widetilde{K}_{12}(t) \delta \label{ineq:diffz}
\end{align}
donde $\widetilde{K}_{12} = K(t) + \widetilde{K}_{11}$

y por \eqref{ineq:diffm}, para $r\ge 0$ y para $0\le t < T$:
\begin{align*}
 |M(r,t) - \widetilde{M}(r,t)| &\underset{\textrm{\eqref{ineq:diffm}, \eqref{ineq:diffz}}}{\le} \widetilde{K}_{8}(t) \delta + \ktilde_{8}(t) ||z-\widetilde{z}||(t) \\
 &\le \ktilde_8(t)\delta + \ktilde_8(t)\ktilde_{12}(t) \delta \\
 &\le \ktilde_{13}(t)\delta
\end{align*}

Basta tomar $\ktilde = \max(\ktilde_{12}, \ktilde_{13})$ para llegar al resultado del teorema.

Nos queda probar que $T = + \infty$ cuando $\delta$ se acerca a $0$.

Sea $T_0 > 0$ y sea $\delta < \frac{r_0}{3 \ktilde(T_0)}$
Entonces para $0 \le t < T$, $0 \le t < T_0$ tenemos que:

\begin{align*}
\frac{2}{3}r_0 \le r_0 - \ktilde(T_0) \delta \le r_0 - \ktilde(t) \delta \le z_j(t) - |\widetilde{z}_j(t) - z_j(t)| \\
\le \widetilde{z}_j(t) \le z_j(t) + |\widetilde{z}_j(t) - z_j(t)|
\le R_0 + U_0t + \ktilde \delta \\
\le R_0 + U_0t + \frac{1}{3}r_0 \le \frac{4}{3}(R_0 + U_0t)
\end{align*}

Si $T < T_0$ esto contravendría que $[0,T[$ sea el intervalo maximal en el que puede resolverse la ecuación diferencial verificando:

\[
\frac{1}{2}r_0 \le \widetilde{z}_j(t) \le 2(R_0 + U_0t)
\]

Tomando $T_0 \rightarrow +\infty$, llegamos a $\delta \rightarrow 0 \Rightarrow T\rightarrow +\infty$
\end{proof}

\begin{corollary}
 Existe una función creciente $\dtilde{K}(t)$ (dependiente solamente de $\phi_0$) verificando que para todo $0 \le t < T$:
 \[
  \Bigg|E - \sum_{i=1}^N M_i\bigg\{\widetilde{z}_i'(t)^2 + L_i^2\widetilde{z}_i'(t)^{-2} + \sum_{j=1}^N \frac{\gamma M_j}{\max(\widetilde{z}_i(t), \widetilde{z}_j(t))}\bigg\} \Bigg| \le \dtilde{K}(t) \delta
 \]
\end{corollary}
\begin{proof}
 Fijamos $(r,u,\alpha) \in S_i, (r',u',\alpha') \in S_j$ y llamamos $z = z(r,u,\alpha,t)$ y $\widetilde{z} = z(r',u',\alpha',t)$. Entonces:
 
 \begin{align*}
 \Bigg|\frac{1}{\max(z,\widetilde{z})} - \frac{1}{\max(\widetilde{z}_i(t), \widetilde{z}_j(t))} \Bigg| &\le \Bigg|\frac{1}{\max(z,\widetilde{z})} - \frac{1}{\max(\widetilde{z}_i(t),\widetilde{z})} \Bigg|\\
 &+ \Bigg|\frac{1}{\max(\widetilde{z}_i(t),\widetilde{z})} - \frac{1}{\max(\widetilde{z}_i(t), \widetilde{z}_j(t))}\Bigg|\\
 &\le \frac{|z-\widetilde{z}_i(t)|}{(r_0/2)^2} + \frac{|\widetilde{z}-\widetilde{z}_j(t)|}{(r_0/2)^2}\\
 &\underset{\textrm{Teorema \ref{th:fundamental}}}{\le} \frac{\ktilde(t)\delta}{(r_0/2)^2} + \frac{\ktilde(t)\delta}{(r_0/2)^2} = \frac{8\ktilde(t)\delta}{r_0^2}
 \end{align*}
donde en la última desigualdad se ha usado (lo hacemos para el segundo término):
\[
  \Bigg|\frac{1}{\max(\widetilde{z}_i(t),\widetilde{z})} - \frac{1}{\max(\widetilde{z}_i(t), \widetilde{z}_j(t))}\Bigg| = \Bigg|\frac{\max(\widetilde{z}_i(t), \widetilde{z}_j(t)) - \max(\widetilde{z}_i(t), \widetilde{z})}{\max(\widetilde{z}_i(t),\widetilde{z}) \max(\widetilde{z}_i(t), \widetilde{z}_j(t))} \Bigg|
\]

Para acotar el denominador basta usar $\widetilde{z}_i(t)^{-1} \le r_0^{-1}$. Para acotar el numerador, lo hacemos discutiendo casos, llamando $a = \max(\widetilde{z}_i(t), \widetilde{z}_j(t))$, y $b= \max(\widetilde{z}_i(t), \widetilde{z})$ 

\begin{enumerate}
 \item Si $a= \widetilde{z}_i(t), b=\widetilde{z}_i(t)$ entonces $a-b = 0$.
 \item Si $a = \widetilde{z}_i(t), b = \widetilde{z}(t)$ nos lleva a $\widetilde{z} \ge \widetilde{z}_i(t)\ge \widetilde{z}_j(t)$.
 \item Si $a = \widetilde{z}_j(t), b = \widetilde{z}_i(t)$ nos lleva a $\widetilde{z}_j(t) \ge \widetilde{z}_i(t) \ge \widetilde{z}$.
 \item El último caso es $a = \widetilde{z}_j(t), b = \widetilde{z}(t)$.
\end{enumerate}
En cualquiera de los casos $|a-b| \le |\widetilde{z} - \widetilde{z}_j(t)|$.

Por un lado:
\begin{align*}
 &\Bigg|\int_{\mathbb{R}^6}\frac{\phi_0(r',u',\alpha')}{\max(z,\widetilde{z})} dv' dx' - \sum_{j=1}^N \frac{M_j}{\max(\widetilde{z}_i(t), \widetilde{z}_j(t))} \Bigg| \\
 &= \Bigg|\sum_{j=1}^N \int_{S_j}\phi_0(r',u',\alpha') \Bigg(\frac{1}{\max(z,\widetilde{z})} - \frac{1}{\max(\widetilde{z}_i(t), \widetilde{z}_j(t))} \Bigg)dv' dx'\Bigg| \\
 &\le \sum_{j=1}^N \int_{S_j} \phi_0(r',u',\alpha') \frac{8\ktilde(t)\delta}{r_0^2} dv' dx' = 8Mr_0^{-2} \ktilde(t) \delta = \dtilde{K}_1(t) \delta
\end{align*}

Por otro lado:
\begin{align*}
&\Bigg|\Bigg(z_t^2 + \frac{(ru \cdot sen(\alpha))^2}{z^2}\Bigg) - \Bigg(\widetilde{z}_i'(t)^2 + \frac{L_i^2}{\widetilde{z}_i(t)^2} \Bigg)\Bigg| \\
&\le |z_t^2 - \widetilde{z}_i'(t)^2| + z^{-2}|(ru \cdot sen(\alpha)^2 - L_i^2| + L_i^2|z^{-2} - \widetilde{z}_i(t)^{-2}| \\
&\le |z_t - \widetilde{z}_i'(t)||z_t +\widetilde{z}_i'(t)| \\
&+ z^{-2}|ru \cdot sen(\alpha) - L_i||ru \cdot sen(\alpha) + L_i| \\
&+ L_i^2 z^{-2} \widetilde{z}_i(t)^{-2} |\widetilde{z}_i(t) + z||\widetilde{z}_i(t) - z|
\end{align*}

Para acotar esta última desigualdad usaremos \eqref{ineq:sins} y el hecho de que:
\[
|z_t + \widetilde{z}_i'(t)| \le |2z_t| + |\widetilde{z}_i'(t) - z_t| \underset{\textrm{Teorema \eqref{th:fundamental}}}{\le} 2U_0 + \ktilde(t) \delta
\]

Así se prueba que: 
\[
 \Bigg|\Bigg(z_t^2 + \frac{(ru sen(\alpha))^2}{z^2}\Bigg) - \Bigg(\widetilde{z}_i'(t)^2 + \frac{L_i^2}{\widetilde{z}_i(t)^2} \Bigg)\Bigg| \le \dtilde{K}_2(t) \delta
\]

Tomamos $\dtilde{K}_3(t) = \max(\dtilde{K}_1(t), \dtilde{K}_2(t))$ y deducimos:
\begin{align*}
&\Bigg|E - \sum_{i=1}^N M_i\bigg\{\widetilde{z}_i'(t)^2 + L_i^2\widetilde{z}_i'(t)^{-2} + \sum_{j=1}^N \frac{\gamma M_j}{\max(\widetilde{z}_i(t), \widetilde{z}_j(t))}\bigg\} \Bigg| \\
&= \Bigg|\sum_{i=1}^N \int_{S_i} \phi_0(r,u,\alpha) \Bigg(z_t^2 + \frac{(ru\cdot sen(\alpha))^2}{z^2} + \gamma \int_{\mathbb{R}^6} \frac{\phi_0(r',u',\alpha')}{\max(z,\widetilde{z})} dv' dx' - \\
&\widetilde{z}_i'(t)^2 - L_i^2\widetilde{z}_i'(t)^{-2} - \gamma \sum_{j=1}^N \frac{M_j}{\max(\widetilde{z}_i(t), \widetilde{z}_j(t))} dv dx \Bigg)\Bigg| \\
&\le \sum_{i=1}^N \int_{S_i} \phi_0(r,u,\alpha) \Bigg\{\Bigg| z_t^2 + \frac{(ru\cdot sen(\alpha))^2}{z^2} -
\widetilde{z}_i'(t)^2 - L_i^2\widetilde{z}_i'(t)^{-2} \Bigg| \\
&+ \Bigg|\int_{\mathbb{R}^6} \frac{\phi_0(r',u',\alpha')}{\max(z,\widetilde{z})} dv' dx' - \sum_{j=1}^N \frac{M_j}{\max(\widetilde{z}_i(t), \widetilde{z}_j(t))} dv dx \Bigg|\Bigg\}\Bigg) \\
&\le \sum_{i=1}^N \int_{S_i} \phi_0(r,u,\alpha) 2\dtilde{K}_3(t)\delta \le \dtilde{K}(t) \delta
\end{align*}


 \end{proof}


\newpage
\begin{thebibliography}{10}
    \expandafter\ifx\csname url\endcsname\relax
    \def\url#1{\texttt{#1}}\fi
    \expandafter\ifx\csname urlprefix\endcsname\relax\def\urlprefix{URL }\fi
    \expandafter\ifx\csname href\endcsname\relax
    \def\href#1#2{#2} \def\path#1{#1}\fi
    
    \bibitem{Schaeffer}
    Discrete approximation of the Poisson-Vlasov system\\
    Jack Shaeffer\\
    Quaterly of applied mathematics\\
    1987
    
    \bibitem{Batt}
    Global symmetric solutions of the initial value problem of stellar dynamics\\
    Jürgen Batt\\
    Journal of Differential Equations\\
    1977
    
    \bibitem{Sonnendrucker}
    Numerical methods for the Vlasov equations\\
    Eric Sonnendrücker\\
    Lecture Notes, Instituto Max-Plank
\end{thebibliography}
	

\end{document}
