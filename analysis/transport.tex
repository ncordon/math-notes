\documentclass[a4paper,10pt]{scrartcl}

% Inclusión de paquetes
\usepackage[utf8]{inputenc}
\usepackage[spanish]{babel}


\usepackage{amsmath}
\usepackage{amssymb}
\usepackage{enumerate}
\usepackage{verbatim}
\usepackage{amsthm}
% Imágenes
\usepackage{graphicx}
\usepackage{float}
\usepackage{tikz}
\usetikzlibrary{arrows}

% Enlaces dentro del documento
\usepackage{hyperref}

% Definiciones
\theoremstyle{definition}
\newtheorem*{mydef}{Definición}
%\newtheorem{mydefn}{Definición}
\newtheorem*{theorem*}{Teorema}
\newtheorem{theorem}{Teorema}
\newtheorem*{lemma*}{Lema}
\newtheorem{lemma}{Lema}
\newtheorem*{fact*}{Proposición}
\newtheorem{fact}{Proposición}
\newtheorem*{corollary*}{Corolario}
\newtheorem{corollary}{Corolario}
\newtheorem*{rmk*}{Nota}
\newtheorem*{eg}{Ejemplo}

% Displaystyle por defecto
\everymath{\displaystyle}

% Comandos
\newcommand{\Referencia}[4]{\indent #1, \textbf{#2}. \textit{#3}, \textit{#4}.\\}
\renewcommand\refname{Referencias}
\renewcommand\contentsname{Contenidos}
\numberwithin{equation}{section}
\setlength{\parindent}{0cm} % Sin sangrías
\setlength{\parskip}{0.3cm}


\title{Aproximación discreta de la ecuación de Vlasov}
\author{
	Ignacio Cordón
}
\date{}

\begin{document}
\maketitle
\begin{center}
    %\includegraphics[width=0.4\textwidth]{./imgs/by-nc-sa.png}
\end{center}
\tableofcontents
\pagebreak

\section{Introducción}
En la naturaleza podemos encontrar 4 estados de la materia: sólido, líquido, gaseoso y plasma. Este último se alcanza con grandes temperaturas a las que los electrones de los átomos abandonan la órbita alrededor del núcleo al que pertenecen, dando lugar a una mezcla de iones, electrones y particulas gobernadas por fuerzas electromagnéticas. Este estado de la materia cobra especial relevancia en procesos como la fusión termonuclear, para la que se necesitan grandes temperaturas.  

La ecuación de \textit{Vlasov} modeliza la evolución de las partículas de un plasma desde un punto de vista estadístico-físico:

\begin{equation}
 \frac{\partial f}{\partial t} + v \frac{\partial f}{\partial x} + \frac{q}{m} (E + v\times B) \cdot \frac{\partial f}{\partial v} = 0
\end{equation}

donde:

\begin{itemize}
 \item $f = f(x,v,t)$ es una función de densidad representando la probabilida de tener una partícula con posición $x \in \mathbb{R}^3$ y velocidad $v \in \mathbb{R}^3$ en un instante de tiempo $t$.
 \item $E$ representa el campo eléctrico.
 \item $B$ representa el campo de inducción magnética.
 \item $F(x,t) = \frac{q}{m}(E + v\times B)$ es la fuerza de Lorentz.
\end{itemize}

El campo electromagnético viene dado por las ecuaciones de Maxwell:

\begin{align*}
 -\frac{1}{c^2} \frac{\partial E}{\partial t} + \nabla \times B = \mu_0 J\\
 \frac{\partial B}{\partial t} + \nabla \times E = 0\\
 \nabla \cdot E = \frac{\rho}{\epsilon_0}\\
 \nabla \cdot B = 0
\end{align*}

donde 
\begin{align*}
\rho(x,t) = q \int f(x,v,t) dv\\
J(x,t) = q \int (x,v,t) v dv
\end{align*}

Cuando el campo magnético es despreciable en comparación con el eléctrico, las ecuaciones de Maxwell quedan reducidas a:

\begin{equation*}
 \nabla \times E = 0, \quad \nabla \cdot E = \frac{\rho}{\epsilon_0}
\end{equation*}

Bajo ciertas condiciones geométricas, $\nabla \times E = 0$ implica que existe un potencial electroestático, $U$, esto es, $E = - \nabla U$, verificando que es solución de la ecuación de Poisson $-\Delta U = \frac{\rho}{\epsilon_0}$ y nuestro modelo se reduce a la ecuación de Vlasov-Poisson:
\begin{align}
 \frac{\partial f}{\partial t} + v \nabla_x f + \frac{q}{m}\nabla_x U \cdot \nabla_v f = 0\\
 \rho(x,t) = q \int f(x,v,t) dv \nonumber\\
 -\Delta U = \frac{\rho}{\epsilon_0} \nonumber
\end{align}


La ecuación de Vlasov-Poisson modela el comportamiento de partículas sometidas a un campo eléctrico sin colisiones:

\begin{align}
\label{eqn:v-poisson}
 \frac{\partial f}{\partial t} + v \nabla_x f + F(x,t) \cdot \nabla_v f = 0\\
 F(x,t) = - \nabla_x U(x,t) \nonumber \\ 
 U(x,t) = \gamma \int \frac{\sigma(y,t)}{|x-y|} dy \nonumber \\
 \sigma(x,t) = \int f(x,v,t) dv \nonumber
\end{align}

donde $f:\mathbb{R}^3 \times \mathbb{R}^3 \times \mathbb{R}^{+}_0 \rightarrow \mathbb{R}$ modela la densidad de partículas en el espacio de fases, $\gamma = -1$ en el caso gravitacional y $+1$ en el caso Coulombiano.

El problema de Cauchy consiste en resolver \eqref{eqn:v-poisson} con condición inicial $f(x,v,0) = f_0(x,v)$.

La existencia de solución global es un problema abierto (está probada la existencia local en ciertos casos particulares).

\section{Preliminares}
Dada $f_0(x,v) = \phi(|x|, |v|, cos^{-1}\left(\frac{x\cdot v}{|x||v|}\right)$ donde $\phi \in C_0^1(]0,+\infty[^2 \times ]0,+\pi[)$. Llamando $r = |x|, u = |v|, \alpha = cos^{-1}\left(\frac{x\cdot v}{|x||v|}\right)$, \cite{Batt} prueba que la solución del problema de Cauchy, $f$, depende únicamente de $r, u, \alpha, t$, esto es:

\begin{equation}
\label{eqn:f=phi}
f(x,v,t) = \phi(r,u,\alpha, t)
\end{equation}

y que además se verifica, llamando:

\begin{align*}
 & M(r,t) = 4\pi \int_0^r \rho(s,t) s^2 ds\\
 & \rho(s, t) = 2\pi \int_0^{\infty} \left(\int_0^\pi \phi(s, u, \alpha, t) sen(\alpha) d\alpha \right)u^2 du
\end{align*}

cumple que:

\begin{align*}
 & U(x,t) = 4\pi\gamma \int_0^{+\infty} \frac{\rho(s,t)}{\max(r,s)} s^2 ds\\
 & F(x,t) = \gamma M(r,t) r^{-3} x\\
\end{align*}

De \eqref{eqn:f=phi} y de \eqref{eqn:v-poisson} deducimos:

\[
\partial_t \phi + v \nabla_x \phi + F(x,t) \cdot \nabla_v \phi = 0\\
\]

donde usando:

\begin{align*}
 v \nabla_x \phi(r, u, \alpha, t) &= v \left(\partial_r \phi \cdot \frac{\partial r}{\partial x} + \partial_\alpha \phi \cdot \frac{\partial \alpha}{\partial x} \right)\\
 &= v\left(\partial_r \phi \frac{x}{|x|} + \partial_\alpha \phi \cdot \frac{-1}{\sqrt{1 - \left(\frac{xv}{|x||v|}\right)^2}} \cdot \frac{1}{|v||x|^2} \cdot\left(v|x| - xv \frac{x}{|x|}\right)\right)\\
 &= \frac{xv}{|x||v|} u \cdot\partial_r \phi - \partial_\alpha \phi \cdot \frac{1}{\sqrt{1 - \left(\frac{xv}{|x||v|}\right)^2}} \cdot \frac{|v|^2|x|^2 - (xv)^2}{|v|^2|x|^2} \frac{|v|}{|x|}\\
 &\underset{\left\{\begin{array}{c} \frac{xv}{|x||v|} = cos(\alpha)\\ \alpha \in ]0,\infty[\end{array}\right.}{=} \partial_r \phi \cdot cos(\alpha) u - \partial_\alpha \phi \cdot sen(\alpha) \frac{u}{r}
\end{align*}

y  por otro lado:
\begin{align*}
F(x,t) \nabla_v \phi(r, u, \alpha, t) &= F(x,t) \left(\partial_u \phi \cdot \frac{\partial u}{\partial v} + \partial_\alpha \phi \cdot \frac{\partial \alpha}{\partial v} \right) \\
&= F(x,t) \left(\partial_u \phi \frac{v}{|v|} - \partial_\alpha \phi \cdot \frac{1}{\sqrt{1 - \left(\frac{xv}{|x||v|}\right)^2}} \cdot \frac{1}{|x||v|^2} \cdot\left(x|v| - xv \frac{v}{|v|}\right)\right) \\
&= |x| \gamma M(r,t) r^{-3} x \cdot \partial_u \phi \frac{v}{|v||x|}\\
&- \gamma M(r,t) r^{-3} x \cdot \partial_\alpha \phi \cdot \frac{1}{\sqrt{1 - \left(\frac{xv}{|x||v|}\right)^2}} \cdot \frac{1}{|x||v|^2} \cdot\left(x|v| - xv \frac{v}{|v|}\right)\\
&= \partial_u \phi \cdot \gamma M(r,t) r^{-2} cos(\alpha) - \partial_\alpha \phi \cdot \gamma M(r,t) r^{-3} \cdot sen(\alpha) \frac{r}{u} \\
&= \partial_u \phi \cdot \gamma M(r,t) r^{-2} cos(\alpha) - \partial_\alpha \phi \cdot \gamma M(r,t) r^{-2} u^{-1} \cdot sen(\alpha)
\end{align*}

Luego:

\begin{equation}
\label{eqn:sph-vlasov}
\partial_t \phi + \underbrace{
    \left(\begin{array}{c}
        cos(\alpha)u\\
        \gamma M(r,t) r^{-2} cos(\alpha)\\
        (-\gamma M(r,t) r^{-2} u^{-1} - ur^{-1})\cdot sen(\alpha)
    \end{array}\right)}_{\mathcal{F}(r,u,\alpha)}
    \cdot \nabla_{(r,u,\alpha)} \phi = 0
\end{equation}

Sabemos que \eqref{eqn:sph-vlasov} tiene como curvas características $X(t; 0, (r,u,\alpha)) = 
\left(\begin{array}{c} 
       z(t; 0, (r, u, \alpha))\\ q(t; 0, (r, u, \alpha))\\ \theta(t; 0, (r, u, \alpha))
      \end{array}\right)$
verificando:

\begin{equation}
\label{eqn:sph-cc}
\partial_t X(t; 0,(r,u,\alpha)) = \mathcal{F}(r,u,\alpha)
\end{equation}

Desde \eqref{eqn:sph-cc} deducimos que $zqsen(\theta)$ es independiente de $t$ (derivando sobre $t$) y también, derivando la primera componente de $X(t)$, y teniendo en cuenta que $\left(\begin{array}{c} z(0)\\ q(0)\\ \theta(0) \end{array}\right) = 
\left(\begin{array}{c} r\\ u\\ \alpha\end{array}\right)$ llegamos a:

\[
 z_{tt} - (ru sen\alpha)^2 \cdot z^{-3} - \gamma M(z,t) z^{-2} = 0
\]

Llamando $\mathcal{C} \subseteq ]0, +\infty[^2 \times ]0, \pi[$ al soporte de $\phi_0$, puede verse fácilmente que existen $r_0, R_0, l_0, L_0, U_0, \rho_0$, dependientes sólo de $\phi_0$, verificando que para todo $(r,u,\alpha,t) \in \mathcal{C} \times ]0, +\infty[$ se cumple:

\begin{align*}
 r_0 \le z(r,u,\alpha,t) \le R_0 + U_0 t\\
 l_o \le ru \le L_0\\
 l_0 \le ru sen(\alpha) 0 = zq sen(\theta) \le L_0\\
 q(r,u,\alpha, t) \le U_0
 \rho(r,t) \le \rho_0
\end{align*}

Además tenemos que para todo $\varphi \in L^{\infty}(]0,+\infty[^2 \times ]0,\pi[)$:

\[
    \int_{\mathbb{R}^3}\int_{\mathbb{R}^3} \varphi(r,u,\alpha) \cdot \phi(r,u,\alpha,t) dv dx = 
    \int_{\mathbb{R}^3}\int_{\mathbb{R}^3} \varphi(z,q,\theta) \cdot \phi_0(r,u,\alpha,t) dv_0 dx_0 = 
\]

donde en la última integral tenemos que $z,q,\theta$ son las curvas características verificando 
\[
    \left(\begin{array}{c} z(0)\\ q(0)\\ \theta(0) \end{array}\right) = 
    \left(\begin{array}{c} x_0\\ v_0\\ \theta=cos^{-1}\left(\frac{x_0 v_0}{|x_0||v_0|}\right) \end{array}\right)
\]

En el caso particular $\varphi = 1$ tendríamos:

\[ 
    M = \int\int \phi_0(r,u,\alpha) dv_0 dx_0 = \int\int \phi(r,u,\alpha,t) dv dx = 4\pi \int_0^{\infty} \rho(s,t) s^2 ds 
\]



\newpage
\begin{thebibliography}{10}
    \expandafter\ifx\csname url\endcsname\relax
    \def\url#1{\texttt{#1}}\fi
    \expandafter\ifx\csname urlprefix\endcsname\relax\def\urlprefix{URL }\fi
    \expandafter\ifx\csname href\endcsname\relax
    \def\href#1#2{#2} \def\path#1{#1}\fi
    
    \bibitem{Schaeffer}
    Discrete approximation of the Poisson-Vlasov system\\
    Jack Shaeffer\\
    Quaterly of applied mathematics\\
    1987
    
    \bibitem{Batt}
    Global symmetric solutions of the initial value problem of stellar dynamics\\
    Jürgen Batt\\
    Journal of Differential Equations\\
    1977
\end{thebibliography}
	

\end{document}
