\documentclass[a4paper,10pt]{scrartcl}

% Inclusión de paquetes
\usepackage[utf8]{inputenc}
\usepackage[spanish]{babel}

\usepackage{dsfont}
\usepackage{amsmath}
\usepackage{amssymb}
\usepackage{enumerate}
\usepackage{verbatim}
\usepackage{amsthm}
% Imágenes
\usepackage{graphicx}
\usepackage{float}
\usepackage{tikz}
\usetikzlibrary{arrows}

% Enlaces dentro del documento
\usepackage{hyperref}
\usepackage{cleveref}

% Definiciones
\theoremstyle{definition}
\newtheorem*{mydef}{Definición}
%\newtheorem{mydefn}{Definición}
\newtheorem*{theorem*}{Teorema}
\newtheorem{theorem}{Teorema}
\newtheorem*{lemma*}{Lema}
\newtheorem{lemma}{Lema}
\newtheorem*{fact*}{Proposición}
\newtheorem{fact}{Proposición}
\newtheorem*{corollary*}{Corolario}
\newtheorem{corollary}{Corolario}
\newtheorem*{rmk*}{Nota}
\newtheorem*{eg}{Ejemplo}


\newcommand{\ktilde}{\widetilde{K}}
\newcommand{\dtilde}[1]{\widetilde{\widetilde{#1}}}
% Displaystyle por defecto
\everymath{\displaystyle}

% Comandos
\newcommand{\Referencia}[4]{\indent #1, \textbf{#2}. \textit{#3}, \textit{#4}.\\}
\renewcommand\refname{Referencias}
\renewcommand\contentsname{Contenidos}
\numberwithin{equation}{section}
\setlength{\parindent}{0cm} % Sin sangrías
\setlength{\parskip}{0.3cm}


\title{Aproximación discreta de la ecuación de Vlasov}
\author{
	Ignacio Cordón
}
\date{}

\begin{document}
\maketitle
\begin{center}
    %\includegraphics[width=0.4\textwidth]{./imgs/by-nc-sa.png}
\end{center}
\tableofcontents
\pagebreak

\section{Introducción}
En la naturaleza podemos encontrar 4 estados de la materia: sólido, líquido, gaseoso y plasma. Este último se alcanza con grandes temperaturas a las que los electrones de los átomos abandonan la órbita alrededor del núcleo al que pertenecen, dando lugar a una mezcla de iones, electrones y particulas gobernadas por fuerzas electromagnéticas. Este estado de la materia cobra especial relevancia en procesos como la fusión termonuclear, para la que se necesitan grandes temperaturas.  

La ecuación de \textit{Vlasov} modeliza la evolución de las partículas de un plasma desde un punto de vista estadístico-físico:

\begin{equation}
 \frac{\partial f}{\partial t} + v \frac{\partial f}{\partial x} + \frac{q}{m} (E + v\times B) \cdot \frac{\partial f}{\partial v} = 0
\end{equation}

donde:

\begin{itemize}
 \item $f = f(x,v,t)$ es una función de densidad representando la probabilida de tener una partícula con posición $x \in \mathbb{R}^3$ y velocidad $v \in \mathbb{R}^3$ en un instante de tiempo $t$.
 \item $E$ representa el campo eléctrico.
 \item $B$ representa el campo de inducción magnética.
 \item $F(x,t) = \frac{q}{m}(E + v\times B)$ es la fuerza de Lorentz.
\end{itemize}

El campo electromagnético viene dado por las ecuaciones de Maxwell:

\begin{align*}
 -\frac{1}{c^2} \frac{\partial E}{\partial t} + \nabla \times B = \mu_0 J\\
 \frac{\partial B}{\partial t} + \nabla \times E = 0\\
 \nabla \cdot E = \frac{\rho}{\varepsilon_0}\\
 \nabla \cdot B = 0
\end{align*}

donde 
\begin{align*}
\rho(x,t) = q \int f(x,v,t) dv\\
J(x,t) = q \int (x,v,t) v dv
\end{align*}

Cuando el campo magnético es despreciable en comparación con el eléctrico, las ecuaciones de Maxwell quedan reducidas a:

\begin{equation*}
 \nabla \times E = 0, \quad \nabla \cdot E = \frac{\rho}{\varepsilon_0}
\end{equation*}

Bajo ciertas condiciones geométricas, $\nabla \times E = 0$ implica que existe un potencial electroestático, $U$, esto es, $E = - \nabla U$, verificando que es solución de la ecuación de Poisson $-\Delta U = \frac{\rho}{\varepsilon_0}$ y nuestro modelo se reduce a la ecuación de Vlasov-Poisson:
\begin{align}
 \frac{\partial f}{\partial t} + v \nabla_x f + \frac{q}{m}\nabla_x U \cdot \nabla_v f = 0\\
 \rho(x,t) = q \int f(x,v,t) dv \nonumber\\
 -\Delta U = \frac{\rho}{\varepsilon_0} \nonumber
\end{align}


La ecuación de Vlasov-Poisson modela el comportamiento de partículas sometidas a un campo eléctrico sin colisiones:

\begin{align}
\label{eqn:v-poisson}
 \frac{\partial f}{\partial t} + v \nabla_x f + F(x,t) \cdot \nabla_v f = 0\\
 F(x,t) = - \nabla_x U(x,t) \nonumber \\ 
 U(x,t) = \gamma \int \frac{\sigma(y,t)}{|x-y|} dy \nonumber \\
 \sigma(x,t) = \int f(x,v,t) dv \nonumber
\end{align}

donde $f:\mathbb{R}^3 \times \mathbb{R}^3 \times \mathbb{R}^{+}_0 \rightarrow \mathbb{R}$ modela la densidad de partículas en el espacio de fases, $\gamma = -1$ en el caso gravitacional y $+1$ en el caso Coulombiano.

El problema de Cauchy consiste en resolver \eqref{eqn:v-poisson} con condición inicial $f(x,v,0) = f_0(x,v)$.

La existencia de solución global es un problema abierto (está probada la existencia local en ciertos casos particulares).

\section{Preliminares}
Dada $f_0(x,v) = \phi(|x|, |v|, \sphericalangle(x,v))$ donde $\phi \in C_0^1(]0,+\infty[^2 \times ]0,+\pi[)$. Llamando $r = |x|, u = |v|, \alpha = \sphericalangle(x,v) = cos^{-1}\left(\frac{x\cdot v}{|x||v|}\right)$, \cite{Batt} prueba que la solución del problema de Cauchy, $f$, depende únicamente de $r, u, \alpha, t$, esto es:

\begin{equation}
\label{eqn:f=phi}
f(x,v,t) = \phi(r,u,\alpha, t)
\end{equation}

y que además se verifica, llamando:

\begin{align*}
 & M(r,t) = 4\pi \int_0^r \rho(s,t) s^2 ds\\
 & \rho(s, t) = 2\pi \int_0^{\infty} \left(\int_0^\pi \phi(s, u, \alpha, t) sen(\alpha) d\alpha \right)u^2 du
\end{align*}

cumple que:

\begin{align*}
 & U(x,t) = 4\pi\gamma \int_0^{+\infty} \frac{\rho(s,t)}{\max(r,s)} s^2 ds\\
 & F(x,t) = \gamma M(r,t) r^{-3} x\\
\end{align*}

De \eqref{eqn:f=phi} y de \eqref{eqn:v-poisson} deducimos:

\[
\partial_t \phi + v \nabla_x \phi + F(x,t) \cdot \nabla_v \phi = 0\\
\]

donde usando:

\begin{align*}
 v \nabla_x \phi(r, u, \alpha, t) &= v \left(\partial_r \phi \cdot \frac{\partial r}{\partial x} + \partial_\alpha \phi \cdot \frac{\partial \alpha}{\partial x} \right)\\
 &= v\left(\partial_r \phi \frac{x}{|x|} + \partial_\alpha \phi \cdot \frac{-1}{\sqrt{1 - \left(\frac{xv}{|x||v|}\right)^2}} \cdot \frac{1}{|v||x|^2} \cdot\left(v|x| - xv \frac{x}{|x|}\right)\right)\\
 &= \frac{xv}{|x||v|} u \cdot\partial_r \phi - \partial_\alpha \phi \cdot \frac{1}{\sqrt{1 - \left(\frac{xv}{|x||v|}\right)^2}} \cdot \frac{|v|^2|x|^2 - (xv)^2}{|v|^2|x|^2} \frac{|v|}{|x|}\\
 &\underset{\left\{\begin{array}{c} \frac{xv}{|x||v|} = cos(\alpha)\\ \alpha \in ]0,\infty[\end{array}\right.}{=} \partial_r \phi \cdot cos(\alpha) u - \partial_\alpha \phi \cdot sen(\alpha) \frac{u}{r}
\end{align*}

y  por otro lado:
\begin{align*}
F(x,t) \nabla_v \phi(r, u, \alpha, t) &= F(x,t) \left(\partial_u \phi \cdot \frac{\partial u}{\partial v} + \partial_\alpha \phi \cdot \frac{\partial \alpha}{\partial v} \right) \\
&= F(x,t) \left(\partial_u \phi \frac{v}{|v|} - \partial_\alpha \phi \cdot \frac{1}{\sqrt{1 - \left(\frac{xv}{|x||v|}\right)^2}} \cdot \frac{1}{|x||v|^2} \cdot\left(x|v| - xv \frac{v}{|v|}\right)\right) \\
&= |x| \gamma M(r,t) r^{-3} x \cdot \partial_u \phi \frac{v}{|v||x|}\\
&- \gamma M(r,t) r^{-3} x \cdot \partial_\alpha \phi \cdot \frac{1}{\sqrt{1 - \left(\frac{xv}{|x||v|}\right)^2}} \cdot \frac{1}{|x||v|^2} \cdot\left(x|v| - xv \frac{v}{|v|}\right)\\
&= \partial_u \phi \cdot \gamma M(r,t) r^{-2} cos(\alpha) - \partial_\alpha \phi \cdot \gamma M(r,t) r^{-3} \cdot sen(\alpha) \frac{r}{u} \\
&= \partial_u \phi \cdot \gamma M(r,t) r^{-2} cos(\alpha) - \partial_\alpha \phi \cdot \gamma M(r,t) r^{-2} u^{-1} \cdot sen(\alpha)
\end{align*}

Luego:

\begin{equation}
\label{eqn:sph-vlasov}
\partial_t \phi + \underbrace{
    \left(\begin{array}{c}
        cos(\alpha)u\\
        \gamma M(r,t) r^{-2} cos(\alpha)\\
        (-\gamma M(r,t) r^{-2} u^{-1} - ur^{-1})\cdot sen(\alpha)
    \end{array}\right)}_{\mathcal{F}(r,u,\alpha)}
    \cdot \nabla_{(r,u,\alpha)} \phi = 0
\end{equation}

Sabemos que \eqref{eqn:sph-vlasov} tiene como curvas características $X(t; 0, (r,u,\alpha)) = 
\left(\begin{array}{c} 
       z(t; 0, (r, u, \alpha))\\ q(t; 0, (r, u, \alpha))\\ \theta(t; 0, (r, u, \alpha))
      \end{array}\right)$
verificando:

\begin{equation}
\label{eqn:sph-cc}
\partial_t X(t; 0,(r,u,\alpha)) = \mathcal{F}(r,u,\alpha)
\end{equation}

Desde \eqref{eqn:sph-cc} deducimos que $zqsen(\theta)$ es independiente de $t$ (derivando sobre $t$) y también, derivando la primera componente de $X(t)$, y teniendo en cuenta que $\left(\begin{array}{c} z(0)\\ q(0)\\ \theta(0) \end{array}\right) = 
\left(\begin{array}{c} r\\ u\\ \alpha\end{array}\right)$ llegamos a:

\begin{equation}
\label{eqn:curves}
 z_{tt} - (ru sen\alpha)^2 \cdot z^{-3} - \gamma M(z,t) z^{-2} = 0
\end{equation}

Llamando $\mathcal{C} \subseteq ]0, +\infty[^2 \times ]0, \pi[$ al soporte de $\phi_0$, puede verse fácilmente que existen $r_0, R_0, l_0, L_0, U_0, \rho_0$, dependientes sólo de $\phi_0$, verificando que para todo $(r,u,\alpha,t) \in \mathcal{C} \times ]0, +\infty[$ se cumple:

\begin{align}
 r_0 \le z(r,u,\alpha,t) \le R_0 + U_0 t \nonumber\\
 l_o \le ru \le L_0 \nonumber\\
 l_0 \le ru sen(\alpha) 0 = zq sen(\theta) \le L_0 \nonumber\\
 q(r,u,\alpha, t) \le U_0 \nonumber \\
 \rho(r,t) \le \rho_0 \label{eqn:zbounds}
\end{align}

Además tenemos que para todo $\varphi \in L^{\infty}(]0,+\infty[^2 \times ]0,\pi[)$:

\[
    \int_{\mathbb{R}^3}\int_{\mathbb{R}^3} \varphi(r,u,\alpha) \cdot \phi(r,u,\alpha,t) dv dx = 
    \int_{\mathbb{R}^3}\int_{\mathbb{R}^3} \varphi(z,q,\theta) \cdot \phi_0(r,u,\alpha,t) dv_0 dx_0 = 
\]

donde en la última integral tenemos que $z,q,\theta$ son las curvas características verificando 
\[
    \left(\begin{array}{c} z(0)\\ q(0)\\ \theta(0) \end{array}\right) = 
    \left(\begin{array}{c} x_0\\ v_0\\ \theta=cos^{-1}\left(\frac{x_0 v_0}{|x_0||v_0|}\right) \end{array}\right)
\]

En el caso particular $\varphi = 1$ tendríamos:

\[ 
    M = \int\int \phi_0(r,u,\alpha) dv_0 dx_0 = \int\int \phi(r,u,\alpha,t) dv dx = 4\pi \int_0^{\infty} \rho(s,t) s^2 ds 
\]

\begin{lemma}[Desigualdad de Gronwall]
Sean $\alpha, \beta, u$ funcones continuas, con $\alpha$ no decreciente, verificando para todo $t\in I$ intervalo:
\[
 u(t) \le \alpha(t) + \int_a^t \beta(s) u(s)
\]

Entonces: $u(t) \le \alpha(t) \cdot exp\left\{\int_a^t \beta(s) ds\right\}$ para todo $t\in I$.
\end{lemma}

\begin{fact}
 Existe una función creciente $K(t) \ge 1$ dependiente sólo de $\phi_0$ verificando que para todo $c_1=(r_1,u_1,\alpha_1), \, c_2=(r_2,u_2,\alpha_2) \in \mathcal{C}$ y para todo $t\ge 0$:
 \[|z(c_1,t) - z(c_2,t)| + |z_t(c_1,t) - z_t(c_2,t)| \le K(t)|c_1 - c_2|\]
 \label{fact:ineq-curves}
\end{fact}

\begin{proof}
 Fijamos $c_1, c_2$ y notamos:
 \begin{align*}
    z_i(t) &= z(t; 0, c_i)\\
    Q(t) &= \sup\{|z_1(s) - z_2(s)| + |z_1'(s) - z_2'(s)|: 0 \le s \le t\}
 \end{align*}
 
 
 Usaremos:
 \begin{align*}
 |r_1 u_1 sen(\alpha_1) - r_2 u_2 sen(\alpha_2)| &= |r_1 u_1 (sen(\alpha_1) - sen(\alpha_2) + sen(\alpha_2)) - (r_2 - r_1 + r_1) u_2 sen(\alpha_2)| \\
 &\le r_1 u_1 |sen(\alpha_1) - sen(\alpha_2)| + r_1|u_1 - u_2| sen(\alpha_2) + u_2|r_1 - r_2| sen(\alpha_2) \\
 &\underset{|sen(\alpha_1) - sen(\alpha_2)| \le |\alpha_1 - \alpha_2|}{\le} L_0|\alpha_1 - \alpha_2| + R_0|u_1 - u_2| + U_0|r_1 - r_2| \le K|c_1 - c_2|
 \end{align*}
 
 Análogamente:
 \[
  |r_1 u_1 cos(\alpha_1) - r_2 u_2 cos(\alpha_2)| \le K|c_1 - c_2|
 \]
 
 Además:
 
\[
 |M(z_1, t) - M(z_2, t)| = \bigg|4\pi \int_{z_1}^{z_2} \rho(s,t) s^2 ds \bigg| \le 4\pi(R_0 + U_0t)^2 \rho_0 |z_1 - z_2|
\]

 
 
 Así, para todo $t\ge 0$ se cumple:
 
 \begin{align*}
  |z_1'' - z_2''| &=
  \bigg|(r_1 u_1 sen(\alpha_1))^2 z_1^{-3} - (r_2 u_2 sen(\alpha_2))^2 z_2^{-3} + \gamma M(z_1, t) z_1^{-2} - \gamma M(z_2, t) z_2^{-2} \bigg| \\
  &= \bigg|(r_1 u_1 sen(\alpha_1))^2 z_1^{-3} - (r_2 u_2 sen(\alpha_2))^2 (z_2^{-3} - z_1^{-3} + z_1^{-3}) \\
  &+ \gamma M(z_1, t) z_1^{-2} - \gamma M(z_2, t) (z_2^{-2} - z_1^{-2} + z_1^{2}) \bigg| \\
  &\le \bigg|(r_1 u_1 sen(\alpha_1)^2 - (r_2 u_2 sen(\alpha_2)^2 \bigg| z_1^{-3} + (r_2 u_2 sen(\alpha_2))^2 \bigg|z_1^{-3} - z_2^{-3}\bigg| \\
  &+ \bigg|M(z_1,t) - M(z_2,t) \bigg|z_1^{-2} + M(z_2, t) \bigg|z_1^{-2} - z_2^{-2}\bigg| \\
  &\le (r_1 u_1 sen(\alpha_1)+ r_2 u_2 sen(\alpha_2))\bigg|r_1 u_1 sen(\alpha_1) - r_2 u_2 sen(\alpha_2)\bigg| r_0^{-3} \\
  &+ L_0^2 z_1^{-3}z_2^{-3} \bigg|z_2^3 - z_1^3 \bigg| + \bigg|M(z_1,t) - M(z_2,t)\bigg| r_0^{-2} + Mz_1^{-2} z_2^{-2} \bigg|z_2^2 - z_1^2\bigg| \\
  &\le 2L_0 r_0^{-3}\bigg|r_1 u_1 sen(\alpha_1) - r_2 u_2 sen(\alpha_2)\bigg|
  + L_0 r_0^{-6}(z_1^2 + z_1 z_2 + z_2^2)\bigg|z_1 - z_2\bigg|\\
  &+ r_0^{-2}\bigg|M(z_1,t) - M(z_2,t)\bigg| + Mr_0^{-4}(z_1 + z_2) \bigg|z_1 - z_2\bigg| \le 2L_0 r_0^{-3} K|c_1 - c_2| \\
  &+ L_0 r_0^{-6} 3(R_0 + U_0t)^2 |z_1 - z_2| + r_0^{-2} 4\pi (R_0 + U_0t)^2 \rho_0 |z_1 - z_2| + Mr_0^{-4} 2(R_0 + U_0t) |z_1 - z_2|\\
  &= K|c_1 - c_2| + K(t)|z_1 - z_2|
 \end{align*}
 
 donde $K(t)$ es una función creciente.
 
 \begin{align*}
  |z_1 - z_2| + |z_1' - z_2'| &= \left|r_1 - r_2 + \int_0^t (z_1' - z_2')\right| + \left|u_1 cos(\alpha_1) - u_2 cos(\alpha_2) + \int_0^t (z_1'' - z_2'') \right|\\
  &\le |r_1 - r_2| + |u_1 cos(\alpha_1) - u_2 cos(\alpha_2)| + \int_0^t \bigg(|z_1' - z_2'| + |z_1'' - z_2''|\bigg) \\
  &\le |r_1 - r_2| + |u_1 cos(\alpha_1) - u_2 cos(\alpha_2)| + \int_0^t \bigg(Q(s) + K|c_1 - c_2| + K(s)Q(s)\bigg) ds \\
  &\le |r_1 - r_2| + |u_1 cos(\alpha_1) - u_2 cos(\alpha_2)| + Kt|c_1 - c_2| + (1 + K(t)) \int_0^t Q(s) ds\\
  &\le \widetilde{K}(t)|c_1 - c_2| + \widetilde{K}(t) \int_0^t Q(s) ds
 \end{align*}

 para cierta $\widetilde{K} \ge 1$ creciente. Pero $\widetilde{K}(t)$ es creciente, luego para todo $0 \le s \le t$:
 \[
    |z_1(s) - z_2(s)| + |z_1'(s) - z_2'(s)|  \le \widetilde{K}(t)|c_1 - c_2| + \widetilde{K}(t) \int_0^t Q(s)
 \]

 Luego para $t\in [0,T]$:
 \[
  Q(t) \le \widetilde{K}(t)|c_1 - c_2| + \widetilde{K}(t) \int_0^t Q(s) \le \widetilde{K}(T)|c_1 - c_2| + \widetilde{K}(T) \int_0^t Q(s)ds
 \]
 y por la desigualdad de Gronwall:
 \[
  Q(t) \le \widetilde{K}(T)|c_1 - c_2| \cdot e^{C(T)t}
 \]
 
 Por consiguiente para $T$ arbitrario:
 \[
  Q(T) \le \widetilde{K}(T)\cdot e^{TC(T)} \cdot |c_1 - c_2|
 \]
 
\end{proof}

\section{Modelo discreto}
Asumamos que $C$, soporte de $\phi_0$, puede escribirse como $C= \bigcup_{i=1}^N S_i$ donde $S_i$ son conexos, y que:
\[
    \delta = \max_{i=1,\ldots, N} diam(S_i) \le \{1,r_0\}
\]

Fijamos $c_i = (r_i, u_i, \alpha_i) \in S_i$ y llamamos:
\begin{itemize}
\item $L_i = r_i u_i sen(\alpha_i)$
\item $M_i = \int_{S_i} \phi_0(r,u,\alpha) dxdv$
\item $z_i = z(c_i)$
\item $\xi(r) = \min\left(\frac{r}{\delta},1\right) \mathds{1}_{]0,\infty[}$
\end{itemize}

$z_i$ cumpliría la ecuación \eqref{eqn:curves}, luego:
\begin{align*}
 z_i'' - L_i^2 z_i^{-3} - \gamma M(z_i, t) z_i^{-2} = 0\\
 z_i(0) = r_i\\
 z_i'(0) = u_i cos(\alpha_i)
\end{align*}

Tomamos las ecuaciones diferenciales, para $i=1, \ldots, N$:
\begin{align}
 \label{eqn:approx-curves}
 \widetilde{z}_i'' - L_i^{2} \widetilde{z}_i^{-3} - \gamma \widetilde{M}(\widetilde{z}_i, t) \widetilde{z}_i^{-2} = 0\\
 \widetilde{z}_i(0) = r_i, \quad \widetilde{z}'_i(0) = u_i cos(\alpha_i) \nonumber
\end{align}
donde $\widetilde{M}(r,t) = \sum_{i=1}^N M_i \xi(r-\widetilde{z}_i(t))$.

Con estas condiciones, \eqref{eqn:approx-curves} y como $r_0 \le r_i \le R_0$, puede tomarse la solución maximal (definida en $]0,T[$), verificando:
\begin{equation}
    \frac{1}{2}r_0 \le \widetilde{z}_i(t) \le 2(R_0 + U_0t)
\label{eqn:barzbounds}
\end{equation}

Además definiremos $\dtilde{M(r,t)} = \sum_{i=1}^N M_i \xi(r-z_i(t))$.
\begin{theorem}
 Existe una función $\widetilde{K}$ verificando que para todo $i = 1, \ldots, N$, para todo $t\in [0,T[$:
 \begin{align*}
  |z(c,t) - \widetilde{z}_i(t)| + |z'(c,t) - \widetilde{z}'_i(t)| &\le \ktilde\delta \qquad \forall c= (r,u,\alpha) \in S_i \\
  |M(s,t) - \widetilde{M}(s,t)| &\le \ktilde\delta \qquad \forall s \ge 0
 \end{align*}
\end{theorem}

\begin{proof}
 Fijamos $t \ge 0, c'\in C$ y llamamos $z = z(c',t)$. Definimos:
 
 \begin{align*}
 J_1 &= \{j: z_j(t) \le z\}\\
 J_2 &= \{j: z_j(t) \le z - K(t)\delta\}
 \end{align*}
 
 Si $c\in S_j$ y $j\in J_1$, por \cref{fact:ineq-curves} se deduce:
 \[
    z(c,t) \le z(c_j,t)| + K(t)|c - c_j| = z_j(t) + K(t) |c - c_j| \le z_j(t) + K(t)\delta \le z + K(t) \delta
 \]
 
 Por tanto:
 \[
    \bigcup_{j\in J_1} S_j \subseteq \{c \in \mathbb{R}^3: z(c,t) \le z + K(t) \delta\} = \overline{A}
 \]
 
 
 \begin{align*}
 \dtilde{M}(z,t) &= \sum_{j = 1}^M M_j \xi(z - z_j(t)) \underset{z >       
 z_j(t) \Rightarrow \xi(z - z_j(t)) = 0}{\le} \sum_{j \in I} M_j\\
  &= \int_{\bigcup_{j\in J_1} S_j} \phi_0(r,u,\alpha) dx dv \le \int_{\overline{A}} \phi_0(r,u,\alpha) dx dv\\
  &= \int_{\{r \le z + K(t)\delta\}} \phi(r,u,\alpha,t) dx dv = M(z + K(t)\delta, t)\\
  & = M(z,t) + \int_{z}^{z+K(t)\delta} 4\pi s^2 \rho(s,t) ds \\ 
  &\le M(z,t) + 4\pi \rho_0 (R_0 + K(t) \delta)^2 K(t) \delta = M(z, t) + \ktilde_1(t) \delta
 \end{align*}

 Por otro lado, si $c \in S_j$ y $z(c,t) \le z - 2K(t) \delta$ entonces por \cref{fact:ineq-curves}, $z_j(t) \le z(c,t) + K(t) \delta \le z - K(t) \delta$. Por tanto:
 \[
  \underline{A} = C \cap \{c\in \mathbb{R}^3:z(c,t) \le z - 2K(t) \delta\} \subseteq \bigcup_{j\in J_2} S_j
 \]
 
 Por tanto:
 \begin{align*}
 \dtilde{M}(z,t) &= \sum_{j = 1}^M M_j \xi(z - z_j(t)) \underset{z_j \le < z - K(t)\delta \implies z - z_j(t) \ge \delta \Rightarrow \xi(z - z_j(t)) = 1}{\ge} \sum_{j \in J_2} M_j\\
  &= \int_{\bigcup_{j\in J_2} S_j} \phi_0(r,u,\alpha) dx dv \ge \int_{\underline{A}} \phi_0(r,u,\alpha) dx dv\\
  &= \int_{\{r \ge z - 2K(t)\delta\}} \phi(r,u,\alpha,t) dx dv = M(\max(0,z-2K(t)\delta), t)\\
  & = M(z,t) - \int_{\max(0,z-2K(t)\delta)}^z 4\pi s^2 \rho(s,t) ds \\ 
  &\ge M(z,t) - 4\pi \rho_0 R_0^2 2K(t)\delta \ge M(z, t) - \ktilde_2(t) \delta
 \end{align*}
 
 Tomando $\ktilde_3 = \max(\ktilde_1, \ktilde_2)$ hemos probado que:
 
 \begin{equation}
 |M(r,t) - \dtilde{M}(r,t)| \le \ktilde_3(t) \delta
 \label{eqn:m-mdtilde1}
 \end{equation}
 para $r \in z(C,t)$, pero queremos el resultado para todo $r\ge 0$.
 
 Observamos que si $r \ge R_0 + U_0 t + \delta$ entonces $M(r,t) = M = \dtilde{M}(r,t)$, y si $0\le r \le r_0$ entonces $M(r,t) = \dtilde{M}(r,t) = 0$. Así, nos restringimos al caso $r_0 \le r \le R_0 + U_0t + 1$. Supongamos que existe $r_1 \in ]r_0, R_0 + U_0t + 1[ \cap z(C, t)^c$ (caso opuesto ya habríamos acabdo la demostración).
 
 Además, $C$ es compacto y $z$ continua, luego podemos considerar s.p.g. $z(C,t) = [r_0, r_2]$, donde $]r_2,r_1[ \cap z(C,t) = \emptyset$.
 Deducimos:
 
 \begin{align}
  M(r_1, t) - M(r_2, t) &= \int_{r_2}^{r_1} 4\pi s^2 \rho(s,t) ds = \int_{\{(x,v): r\in ]r_2, r_1[\}} \phi(r,u,\alpha,t) dv dx \nonumber\\
  &= \int_{\{c\in\mathbb{R}^3: z(c,t) \in ]r_2, r_1[\}} \phi_0(c) dv dx
  \underset{\{c: z(c,t) \in ]r_2,r_1[\} \cap C = \emptyset}{=} 0
  \label{eqn:mdiff}
 \end{align}
 
 Si $z_j(t) = z_j(c_j,t) \in ]r_2 - \delta, r_2]$ y dado $c'\in S_j$, por \cref{fact:ineq-curves} deducimos:
 \[
 -K(t)\delta -\delta \le z(c',t) - z_j(t) - \delta \le z(c',t) - (r_2 - \delta) - \delta = z(c',t) - r_2 \le z(c',t) - z_j(t) \le K(t)\delta
 \]
 
 Llamamos:
 \[
 J_3 = \{j: z_j(t) \in ]r_2 - \delta, r_2]\}
 \]
 
 De $z_j(t) \notin ]r_2,r_1[$ y usando $r_1 > r_2$ deducimos que si $z_j(t) \le r_2 - \delta$ entonces $\xi(r_1 - z_j) = \xi(r_2 - z_j) = 1$; si $z_j(t) > r_1$ entonces $\xi(r_1 - z_j) = \xi(r_2 - z_j) = 0$ y si $z_j \in ]r_2 - \delta, r_2[$ entonces $\xi(r_1 - z_j) > \xi(r_2 - z_j)$. Juntando toda esta información:
 \begin{align}
 &0 \le \dtilde{M}(r_1,t) - \dtilde{M}(r_2,t) = \sum_{j=1}^N M_j \bigg\{\xi(r_1 - z_j(t)) - \xi(r_2 - z_j(t))\bigg\} \le \sum_{j in J_3} M_j \nonumber\\ 
 &=\int_{\cup_{j\in J_3} S_j} \phi_0(r,u,\alpha) dv dx \le \int_{\{c : z(c,t) - r_2| \le \delta + K(t) \delta\}} \phi_0(c)\\
 &= \int_{\{(x,v): ||x| - r_2| \le \delta + K(t) \delta\}} \phi(c,t) dv dx = \int_{max(0, r_2 - \delta - K(t)\delta}^{r_2 + \delta + K(t) \delta} 4\pi s^2 \rho(s,t) ds \nonumber\\
 &\le 4\pi (r_2 + 1 + K(t))^2 \rho_0 2(1+ K(t))\delta \nonumber\\
 &\le 4\pi (R_0 + U_0t + 2 + K(t))^2 \rho_0 2 (1 + K(t))\delta \le \ktilde_4(t)\delta
 \label{eqn:mtildediff}
 \end{align}

 Como $r_2 \in z(C,t)$, entonces $|M(r_2,t) - \dtilde{M}(r_2,t)| \le \ktilde_3(t) \delta$ por \eqref{eqn:m-mdtilde1}.
 \end{proof}
 
 En definitiva:
 \begin{align}  
|M(r_1,t) - \dtilde{M}(r_1,t)| &= |M(r_1,t) - M(r_2,t) + M(r_2,t) - \dtilde{M}(r_2,t) + \dtilde{M}(r_2,t) - \dtilde{M}(r_1,t)| \nonumber \\
&\le \underbrace{|M(r_1,t) - M(r_2,t)|}_{\textrm{\eqref{eqn:mdiff}}} + |M(r_2,t) - \dtilde{M}(r_2,t)| + \underbrace{|\dtilde{M}(r_2,t) - \dtilde{M}(r_1,t)|}_{\textrm{\eqref{eqn:mtildediff}}} \nonumber \\
&\le 0 + \ktilde_3(t) \delta + \ktilde_4(t) \delta \label{eqn:m-mdtilde2}
 \end{align}

 Hemos probado, por \eqref{eqn:m-mdtilde1} y  \eqref{eqn:m-mdtilde2}
 \begin{equation}
  |M(r,t) - \dtilde{M}(r,t)| \le \ktilde_5(t) \delta, \qquad r\ge 0, t\ge 0
  \label{eqn:m-mdtilde}
 \end{equation}
 donde $\ktilde_5 = \ktilde_3 + \ktilde_4$

 Sea $0\le t < T$ y definimos $||z-\widetilde{z}||(t) = \sup_{0\le s \le t} \max_{1\le i\le N} |z_i(s) - \widetilde{z}_i(s)|$.
 
 Fijamos $r\ge 0, t\in [0, T[$. Si $z_j(t) \ge r + ||z - \widetilde{z}||(t) \ge r + ||z - \widetilde{z}||(t) - \delta$, entonces:
 \begin{align*}
  r - z_j(t) \le 0 \Rightarrow \xi(r - z_j(t)) = 0\\
  r - \widetilde{z}_j(t) \le r - (z_j(t) - ||z-\widetilde{z}||(t)) \le 0 \Rightarrow \xi(r - \widetilde{z}_j(t)) = 0
 \end{align*}
 
 Análogamente, si $z_j(t) \le r - ||z - \widetilde{z}||(t) - \delta$, entonces:
  \begin{align*}
  \xi(r - z_j(t)) = 1\\
  \xi(r - \widetilde{z}_j(t)) = 1
 \end{align*}
 
 Como $0 \le \xi(r - z_j(t)) \le 1, 0 \le \xi(r - \widetilde{z}_j(t)) \le 1$ entonces $|\xi(r - \widetilde{z}_j(t)) - \xi(r - z_j(t))| \le 1$. Llamaremos $J_4 = \{j: |r-z_j(t)| \le ||z - \widetilde{z}||(t) + \delta\}$.  Si $j\in J_4, c' \in S_j$, entonces, por \cref{fact:ineq-curves}:
 \begin{align*}
  |z(c',t) - r| &= |z(c',t) - z_j(t) + z_j(t) - r| \le |z(c',t) - z_j(t)| + |z_j(t) - r| \le ||z-\widetilde{z}||(t) + \delta + K(t)\delta\\
  \bigcup_{j\in J_4} S_j &\subseteq \{c\in \mathbb{R}^3: |z(c,t) - r| \le ||z-\widetilde{z}||(t) + \delta + K(t)\delta\} = \underline{B}
 \end{align*}

 Se desprende, llamando $\varepsilon(t) = ||z-\widetilde{z}||(t) + \delta + K(t)\delta \underset{\textrm{\eqref{eqn:barzbounds}, \eqref{eqn:zbounds}}}{\le} 3(R_0 + U_0 t) + 1 + K(t)$:
 \begin{align*}
 |\widetilde{M}(r,t) - \dtilde{M}(r,t)| &= \left|\sum_{j=1}^N \xi(r - \widetilde{z}_j(t)) - \xi(r - z_j(t)) \right| \le \sum_{j\in J_4} M_j
 \le \int_{\bigcup_{j\in J_4} S_j} \phi_0(c) dv dx\\
 &\le \int_{\underline{B}} \phi_0(c) dv dx
 = \int_{\{(x,v):||x| - r| < \varepsilon(t)\}} \phi(r,u,\alpha.t) dv dx \\
 &= \int_{\max(0,r-\varepsilon(t))}^{r+\varepsilon(t)} 4\pi s^2 \rho(s,t) ds \le 4\pi(r + \varepsilon(t))^2 \rho_0 2\varepsilon(t)
 \end{align*}
 
 Si $r > 2(R_0 + U_0t) + 1$ entonces $\widetilde{M}(r,t) = \dtilde{M}(r,t) = M$. Si $r \le 2(R_0 + U_0t) + 1$ entonces:
 \begin{align}
  |\widetilde{M}(r,t) - \dtilde{M}(r,t)| &\le 4\pi\bigg\{5(R_0 + U_0t) + 2 + K(t)\bigg\}^2 \rho_0 2\epsilon(t) \nonumber\\
  &\le \ktilde_6(t)\bigg\{||z-\widetilde{z}||(t) + \delta + K(t)\delta\bigg\} 
  \nonumber \\
  &\le \ktilde_7(t)||z - \widetilde{z}||(t) + \ktilde_7(t)\delta \label{eqn:barbarmbound}
 \end{align}
 
 Deducimos:
 \begin{align}
  |M(r,t) - \widetilde{M}(r,t)| &= |M(r,t) - \dtilde{M}(r,t) + \dtilde{M}(r,t) - \widetilde{M}(r,t)| \nonumber\\
  &\le |M(r,t) - \dtilde{M}(r,t)| + |\dtilde{M}(r,t) - \widetilde{M}(r,t)| \nonumber\\
  &\underset{\textrm{\eqref{eqn:m-mdtilde}, \eqref{eqn:barbarmbound}}}{\le} \ktilde_7(t)||z - \widetilde{z}||(t) + (\ktilde_5(t)+\ktilde_7(t))\delta \nonumber\\
  &\le \ktilde_8(t)||z - \widetilde{z}||(t) + \ktilde_8(t)\delta
 \end{align}
donde $\ktilde_8 = \max(\ktilde_7, \ktilde_5 + \ktilde_7)$

\newpage
\begin{thebibliography}{10}
    \expandafter\ifx\csname url\endcsname\relax
    \def\url#1{\texttt{#1}}\fi
    \expandafter\ifx\csname urlprefix\endcsname\relax\def\urlprefix{URL }\fi
    \expandafter\ifx\csname href\endcsname\relax
    \def\href#1#2{#2} \def\path#1{#1}\fi
    
    \bibitem{Schaeffer}
    Discrete approximation of the Poisson-Vlasov system\\
    Jack Shaeffer\\
    Quaterly of applied mathematics\\
    1987
    
    \bibitem{Batt}
    Global symmetric solutions of the initial value problem of stellar dynamics\\
    Jürgen Batt\\
    Journal of Differential Equations\\
    1977
\end{thebibliography}
	

\end{document}
